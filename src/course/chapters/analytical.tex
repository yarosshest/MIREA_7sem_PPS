\chapter{Аналитический раздел}

\section{Диаграмма Ганта разработки ПС}
Диаграмма Ганта представляет собой визуальное отображение задач
и временных рамок, необходимых для разработки системы.


\begin{image}
	\includegrph[scale=0.4]{Рекомендательная система фильмов1.png}
	\caption{Диаграмма Ганта}
	\label{fig:gant:1}
\end{image}

Вторая часть.

\begin{image}
	\includegrph[scale=0.4]{Рекомендательная система фильмов2.png}
	\caption{Диаграмма Ганта}
	\label{fig:gant:2}
\end{image}


\section{Обоснование средств разработки программного решения}

\subsection{Язык программирования}
\textbf{Python} — используется для серверной части системы, которая отвечает за обработку данных и реализацию алгоритмов рекомендации. Python поддерживает множество библиотек для машинного обучения и работы с базами данных, таких как CatBoost и SqlAlchemy.

\textbf{Kotlin} — основной язык программирования для Android-приложений, обеспечивающий современный и эффективный подход к разработке мобильных интерфейсов. Kotlin, благодаря своей лаконичности и полной совместимости с Java, позволяет создавать стабильные и поддерживаемые приложения.

\subsection{Алгоритм машинного обучения}
\textbf{CatBoost} — это библиотека градиентного бустинга, используемая для построения дерева решений в системе рекомендаций фильмов. CatBoost эффективно работает с категориальными данными и предлагает высокую точность предсказаний при небольшом времени обучения, что делает его идеальным выбором для рекомендательной системы.

\subsection{Интеграционная среда разработки}
\textbf{Pycharm} — это мощная IDE, используемая для разработки серверной части приложения на Python. Она поддерживает отладку кода, тестирование и управление зависимостями, что помогает оптимизировать процесс разработки рекомендательной системы.

\textbf{Android Studio} — основная интегрированная среда разработки для Android-приложений, которая поддерживает Kotlin и Java. С её помощью создается пользовательский интерфейс мобильного приложения, которое взаимодействует с сервером через Rest API.

\subsection{Фреймворк для серверной части}
\textbf{FastAPI} — это современный и высокопроизводительный веб-фреймворк для Python, используемый для создания Rest API серверной части системы. FastAPI обеспечивает быструю разработку и лёгкую масштабируемость, что делает его идеальным выбором для высоконагруженных приложений.

\subsection{База данных}
\textbf{PostgreSQL} — это реляционная база данных, используемая для хранения данных о фильмах и предпочтениях пользователей. PostgreSQL поддерживает работу с большими объемами данных и обладает высоким уровнем производительности, что делает её подходящей для рекомендательных систем.

\subsection{ORM для работы с базой данных}
\textbf{SQLAlchemy} — это мощная ORM-библиотека для Python, которая используется для взаимодействия с базой данных Postgres. Она упрощает работу с базой данных, обеспечивая легкую интеграцию с Python-кодом, и предоставляет удобные инструменты для работы с запросами.

\subsection{Фреймворк для тестирования}
\textbf{Pytest} предоставляет удобные инструменты для написания и выполнения модульных и интеграционных тестов в Python. Он поддерживает автоматизацию тестирования и является важным компонентом для обеспечения стабильности серверной части системы.

\subsection{Система управления версиями}
\textbf{Git} — используется для контроля версий как серверной части, так и мобильного приложения. Интеграция с GitHub или GitLab позволяет эффективно отслеживать изменения кода, организовать работу команды и автоматизировать процессы CI/CD.

\section{Матрица рисков}

Таблица рисков помогает заранее определить
и подготовиться к потенциальным проблемам,
которые могут возникнуть в процессе разработки.

\begin{table}[h]
	\centering
	\caption{Перечень основных рисков проекта}
	\label{tab:risk_table}
	\begin{tabular}{|p{3cm}|p{2.5cm}|p{2.5cm}|p{3cm}|p{4cm}|}
		\hline
		\textbf{Название риска} & \textbf{Послед - ствия} & \textbf{Оценка риска} & \textbf{Стратегия реагирования} & \textbf{Мероприятия в рамках стратегии} \\ \hline
		Задержка поставки требований & Срыв сроков проекта & Высокая & Принятие & Создать резерв времени, ускорить сбор требований \\ \hline
		Недостаточная квалификация команды & Низкое качество продукта & Средняя & Снижение & Организовать обучение, привлечь опытных специалистов \\ \hline
		Отказ оборудования & Остановка разработки & Низкая & Принятие & Обеспечить резервное оборудование, регулярное резервирование данных \\ \hline
		Изменение требований заказчиком & Увеличение объёма работ & Высокая & Снижение & Ввести процесс управления изменениями, согласовать ТЗ \\ \hline
		Риск безопасности данных & Утечка информации & Средняя & Принятие & Внедрить меры информационной безопасности, обучить персонал \\ \hline
	\end{tabular}
\end{table}
\clearpage

\section{Архитектура программного решения}

Основной принцип архитектуры системы рекомендации
--- \textbf{модульность}.
Программа состоит из отдельных компонентов (модулей),
которые отвечают за конкретные функции:
парсинг, рекомендация, основная логика и мобильное приложение.
Это позволяет легко изменять или добавлять новые функции
без необходимости переписывать всю систему.\par
Также в программном продукте используется принцип
\textbf{разделения ответственности}
(Separation of Concerns или SoC), где каждому модулю назначена
своя ответственность, что снижает взаимозависимость между компонентами.
Это делает систему более поддерживаемой и упрощает тестирование и отладку.

\subsection{Архитектурная диаграмма}

Модель C4 описывает архитектуру системы на четырёх уровнях:
контекста, контейнеров, компонентов и кода.
На основе этой модели можно создать следующие диаграммы.


\subsubsection{Диаграмма контекста системы}
В модели C4 диаграмма контекста системы (System Context Diagram)
представляет собой самый высокий уровень абстракции системы,
показывая взаимодействие системы с внешними пользователями (акторами)
и системами.
Она дает общее представление о том,
кто и как взаимодействует с системой,
не углубляясь в детали внутренней реализации.

\begin{image}
	\includegrph[scale=0.1]{С41}
	\caption{Диаграмма контекста системы}
	\label{fig:c4:system:context1}
\end{image}

Пользователь взаимодействует с системой через интерфейс мобильного приложения для авторизации, регистрации, просмотра и
оценке фильмов.\par
Кинопоиск является источником данных о фильмах для системы.
\clearpage

\subsubsection{Диаграмма контейнеров}

Диаграмма контейнеров представляет следующий уровень детализации модели C4.
Она показывает внутреннюю структуру системы,
отображая основные контейнеры (программные компоненты) системы,
такие как приложения, базы данных, внешние системы, и их взаимодействия.

\begin{image}
	\includegrph[scale=0.1]{С42}
	\caption{Диаграмма контейнеров}
	\label{fig:c4:container2}
\end{image}


\subsubsection{Диаграмма компонентов}

Диаграмма компонентов модели C4 детализирует
каждый контейнер на более глубоком уровне,
отображая его внутренние программные компоненты и взаимодействия между ними.
Это позволяет увидеть, как функционирует каждая часть контейнера
и какие компоненты обеспечивают выполнение основных функций.

\begin{image}
	\includegrph[scale=0.1]{С43}
	\caption{Диаграмма компонентов}
	\label{fig:c4:components3}
\end{image}

\subsection{Масштабируемость системы}

Масштабируемость системы заключается в её способности обрабатывать все большее кол-во фильмов и пользователей.

\subsubsection{Вертикальная масштабируемость}

При увеличении мощности оборудования
(например, процессора и оперативной памяти)
программа сможет обрабатывать большее кол-во запросов о рекоминдации.

\subsubsection{Масштабируемость за счёт модульности}

Каждый компонент программы
может быть расширен или заменён,
что позволит легко добавлять новые возможности
без изменения существующего кода.

\subsection{Описание инструментов для каждого компонента архитектуры}


\subsubsection{БД}
Компонент взаимодействия с БД через ORM SQLAlhemy, через асинзронные вызовы.
SQLAlchemy — это популярная библиотека для языка программирования Python, предназначенная для работы с базами данных.
Один из ключевых компонентов SQLAlchemy — это ORM (Object-Relational Mapping), которая обеспечивает удобный и абстрактный доступ к данным реляционных баз данных, таких как PostgreSQL, MySQL, SQLite, Oracle и другие.
\subsubsection{API}
Компонент с основной логикой системы , обеспечивает взаимодействие основных компонентов.
Формат хранения данных авторизации пользователей ,
в системе компонент JWT производит авторизацию пользователей и
передает информацию о них дальше.
FastAPI — это современный фреймворк для создания web-приложений на Python, который специализируется на разработке быстрых
и эффективных RESTful API. Он построен поверх асинхронного сервера ASGI (например, Uvicorn или Hypercorn) и использует последние возможности языка Python, такие как аннотации типов, что позволяет автоматически генерировать документацию API и реализовать быструю валидацию данных.
\subsubsection{Система рекоминдации}
Создает рекомендации на основе оценок пользователя.
CatBoost — это высокопроизводительная библиотека машинного обучения, разработанная компанией Яндекс.
Основным назначением CatBoost является решение задач классификации и регрессии с помощью метода градиентного бустинга по решающим деревьям.
CatBoost известен своей эффективностью, скоростью работы и качеством предсказаний, особенно на данных со значительным количеством категориальных признаков.
\subsubsection{Сборщик}
Собирает информацию о фильмах с сайтов.
Beautiful Soup — это популярная библиотека на Python для парсинга HTML и XML документов.
Её основная задача — упростить извлечение данных из веб-страниц, что делает её незаменимой при веб-скрейпинге и анализе структурированных документов.
\section{Модель базы данных}

\subsection{Логическая модель базы данных}


Логическая модель данных в данной системе представляет собой объектно-реляционную структуру,
охватывающую основные сущности и их взаимосвязи, включая \textbf{Product}, \textbf{Distance}, \textbf{Vector}, \textbf{Lemma}, \textbf{Attribute}, \textbf{User} и \textbf{Rate}.
Эти объекты необходимы для хранения и обработки данных, связанных с продуктами, их характеристиками и оценками пользователей.э

\textbf{Product} — это основная модель, представляющая продукт в системе. Каждый продукт может иметь набор атрибутов, векторное представление, лемму и пользовательские оценки.

Атрибуты класса:
\begin{itemize}
	\item \textbf{id:} идентификатор продукта (первичный ключ);
	\item \textbf{name:} имя продукта;
	\item \textbf{photo:} ссылка на фотографию продукта;
	\item \textbf{description:} описание продукта.
\end{itemize}

Связи:
\begin{itemize}
	\item \textbf{distance:} список расстояний (\textbf{Distance}), связанных с продуктом;
	\item \textbf{attribute:} список атрибутов (\textbf{Attribute}), связанных с продуктом;
	\item \textbf{vector:} векторное представление (\textbf{Vector});
	\item \textbf{lemma:} лемма (\textbf{Lemma});
	\item \textbf{rates:} список оценок (\textbf{Rate}), связанных с продуктом.
\end{itemize}

\textbf{Distance} представляет расстояние между двумя продуктами.

Атрибуты класса:
\begin{itemize}
	\item \textbf{id:} идентификатор расстояния (первичный ключ);
	\item \textbf{product\_f\_id:} идентификатор первого продукта;
	\item \textbf{product\_s\_id:} идентификатор второго продукта;
	\item \textbf{distance:} числовое значение расстояния.
\end{itemize}

\textbf{Vector} описывает векторное представление продукта.

Атрибуты класса:
\begin{itemize}
	\item \textbf{id:} идентификатор вектора (первичный ключ);
	\item \textbf{product\_id:} идентификатор продукта;
	\item \textbf{vector:} сериализованное представление вектора.
\end{itemize}

Связь:
\begin{itemize}
	\item \textbf{product:} связь с сущностью \textbf{Product}.
\end{itemize}

\textbf{Lemma} представляет лемматизированное значение, связанное с продуктом.

Атрибуты класса:
\begin{itemize}
	\item \textbf{id:} идентификатор леммы (первичный ключ);
	\item \textbf{product\_id:} идентификатор продукта;
	\item \textbf{lemma:} текстовое значение леммы.
\end{itemize}

Связь:
\begin{itemize}
	\item \textbf{product:} связь с сущностью \textbf{Product}.
\end{itemize}

\textbf{Attribute} хранит характеристики продукта.

Атрибуты класса:
\begin{itemize}
	\item \textbf{id:} идентификатор атрибута (первичный ключ);
	\item \textbf{product\_id:} идентификатор продукта;
	\item \textbf{name:} имя атрибута;
	\item \textbf{value\_type:} тип значения атрибута (опционально);
	\item \textbf{value:} значение атрибута;
	\item \textbf{value\_description:} описание значения (опционально).
\end{itemize}


\textbf{User} описывает пользователя системы.

Атрибуты класса:
\begin{itemize}
	\item \textbf{id:} идентификатор пользователя (первичный ключ);
	\item \textbf{name:} имя пользователя;
	\item \textbf{password:} пароль пользователя.
\end{itemize}

Связь:
\begin{itemize}
	\item \textbf{rates:} список оценок (\textbf{Rate}), сделанных пользователем.
\end{itemize}


\textbf{Rate} представляет оценку продукта, оставленную пользователем.

Атрибуты класса:
\begin{itemize}
	\item \textbf{user\_id:} идентификатор пользователя (первичный ключ);
	\item \textbf{product\_id:} идентификатор продукта (первичный ключ);
	\item \textbf{rate:} булевое значение оценки.
\end{itemize}

Связи:
\begin{itemize}
	\item \textbf{user:} связь с сущностью \textbf{User};
	\item \textbf{product:} связь с сущностью \textbf{Product}.
\end{itemize}

Логическая модель данный представлена ER-диаграммой на рисунке \ref{fig:er}.


\begin{image}
	\includegrph{er.png}
	\caption{ER-диаграмма}
	\label{fig:er}
\end{image}

\subsection{Словарь данных}

На первом этапе проектирования базы данных необходимо собрать сведения о предметной области, в том числе о назначении,
способах использования и охране структуры данных, а по мере развития проекта осуществлять централизованное накопление
информации о концептуальной, логической, внутренней и внешних моделях данных.
Словарь данных является как раз тем средством, которое позволяет при проектировании, эксплуатации и развитии базы
данных поддерживать и контролировать информацию о данных.

\begin{longtable}{|p{3.5cm}|p{5cm}|p{5cm}|}
	\caption{\leftline{Словарь данных Product}} \\
	\hline
	\textbf{Наименование элемента}
	& \textbf{Определение (предназначение)}
	& \textbf{Тип} \\ \hline
	\endhead
	\textbf{id} & Идентификатор продукта (первичный ключ) & Целое \\ \hline
	\textbf{name} & Имя продукта & Строка \\ \hline
	\textbf{photo} & Ссылка на фотографию продукта & Строка \\ \hline
	\textbf{description} & Описание продукта & Строка \\ \hline
\end{longtable}

\begin{longtable}{|p{3.5cm}|p{5cm}|p{5cm}|}
	\caption{\leftline{Словарь данных Distance}} \\
	\hline
	\textbf{Наименование элемента}
	& \textbf{Определение (предназначение)}
	& \textbf{Тип} \\ \hline
	\endhead
	\textbf{id} & Идентификатор расстояния (первичный ключ) & Целое \\ \hline
	\textbf{product\_f\_id} & Идентификатор первого продукта & Целое \\ \hline
	\textbf{product\_s\_id} & Идентификатор второго продукта & Целое \\ \hline
	\textbf{distance} & Значение расстояния между продуктами & Вещественное \\ \hline
\end{longtable}

\begin{longtable}{|p{3.5cm}|p{5cm}|p{5cm}|}
	\caption{\leftline{Словарь данных Vector}} \\
	\hline
	\textbf{Наименование элемента}
	& \textbf{Определение (предназначение)}
	& \textbf{Тип} \\ \hline
	\endhead
	\textbf{id} & Идентификатор вектора (первичный ключ) & Целое \\ \hline
	\textbf{product\_id} & Идентификатор связанного продукта & Целое \\ \hline
	\textbf{vector} & Векторное представление продукта & Сериализованный \\ \hline
\end{longtable}

\begin{longtable}{|p{3.5cm}|p{5cm}|p{5cm}|}
	\caption{\leftline{Словарь данных Lemma}} \\
	\hline
	\textbf{Наименование элемента}
	& \textbf{Определение (предназначение)}
	& \textbf{Тип} \\ \hline
	\endhead
	\textbf{id} & Идентификатор леммы (первичный ключ) & Целое \\ \hline
	\textbf{product\_id} & Идентификатор связанного продукта & Целое \\ \hline
	\textbf{lemma} & Лемматизированное значение & Строка \\ \hline
\end{longtable}

\begin{longtable}{|p{3.5cm}|p{5cm}|p{5cm}|}
	\caption{\leftline{Словарь данных Attribute}} \\
	\hline
	\textbf{Наименование элемента}
	& \textbf{Определение (предназначение)}
	& \textbf{Тип} \\ \hline
	\endhead
	\textbf{id} & Идентификатор атрибута (первичный ключ) & Целое \\ \hline
	\textbf{product\_id} & Идентификатор связанного продукта & Целое \\ \hline
	\textbf{name} & Имя атрибута & Строка \\ \hline
	\textbf{value\_type} & Тип значения атрибута & Строка \\ \hline
	\textbf{value} & Значение атрибута & Строка \\ \hline
	\textbf{value\_description} & Описание значения атрибута & Строка \\ \hline
\end{longtable}

\begin{longtable}{|p{3.5cm}|p{5cm}|p{5cm}|}
	\caption{\leftline{Словарь данных User}} \\
	\hline
	\textbf{Наименование элемента}
	& \textbf{Определение (предназначение)}
	& \textbf{Тип} \\ \hline
	\endhead
	\textbf{id} & Идентификатор пользователя (первичный ключ) & Целое \\ \hline
	\textbf{name} & Имя пользователя & Строка \\ \hline
	\textbf{password} & Пароль пользователя & Строка \\ \hline
\end{longtable}

\begin{longtable}{|p{3.5cm}|p{5cm}|p{5cm}|}
	\caption{\leftline{Словарь данных Rate}} \\
	\hline
	\textbf{Наименование элемента}
	& \textbf{Определение (предназначение)}
	& \textbf{Тип} \\ \hline
	\endhead
	\textbf{user\_id} & Идентификатор пользователя (первичный ключ, внешний ключ) & Целое \\ \hline
	\textbf{product\_id} & Идентификатор продукта (первичный ключ, внешний ключ) & Целое \\ \hline
	\textbf{rate} & Оценка продукта (булевое значение) & Булево \\ \hline
\end{longtable}


\subsection{Матрица доступа и роли}

\subsubsection{Определение ролей в системе}

В системе предусмотрена две роли: \textbf{Пользователь}, \textbf{гость}.
Гость это не зарегистрированный пользователь.

\clearpage

\subsubsection{Матрица доступа}

Матрица доступа для ролей пользователь и гость по отношению к объектам,
использующимся в логической модели:

\begin{longtable}{|p{4cm}|p{4cm}|p{4cm}|}
	\caption{\leftline{Матрица доступа}} \\
	\hline
	\textbf{Таблица} & \textbf{Пользователь} & \textbf{Гость} \\
	\hline
	\endhead

	\textbf{Product} &
	Чтение &
	Чтение \\ \hline

	\textbf{Distance} &
	Нет доступа &
	Нет доступа \\ \hline

	\textbf{Vector} &
	Нет доступа &
	Нет доступа \\ \hline

	\textbf{Lemma} &
	Нет доступа &
	Нет доступа \\ \hline

	\textbf{Attribute} &
	Чтение &
	Чтение \\ \hline

	\textbf{User} &  Чтение \par Обновление &  Нет доступа \\ \hline

	\textbf{Rate} & Создание \par Чтение & Нет доступа \\ \hline
\end{longtable}