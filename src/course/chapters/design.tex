\chapter{Проектный раздел}

\section{Дерево функций}

Дерево зависимостей функций можно построить на основе взаимосвязанных
шагов процесса составления рекомендаций\rref{fig:funcTree}.

\begin{image}
	\includegrph{funcTree}
	\caption{Дерево зависимостей функций}
	\label{fig:funcTree}
\end{image}

Это дерево функций показывает зависимость между этапами работы программы
и связанными с ними подфункциями.

\section{Логика работы программного решения}

\subsection{Диаграмма прецедентов}

Диаграмма прецедентов (Use case diagram) --- это диаграмма поведения,
на которой показано множество прецедентов и актёров,
а также отношения между ними. Она применяется для моделирования вида системы
с точки зрения внешнего наблюдателя.

Диаграмма прецедентов представляет собой визуальное представление
взаимодействий между пользователем и системой, описывающее,
какие действия (прецеденты) может выполнить пользователь \rref{fig:use:case}.

\begin{image}
	\includegrph[scale=0.27]{usecse}
	\caption{Диаграмма прецедентов}
	\label{fig:use:case}
\end{image}

\subsection{Диаграмма последовательности}

Диаграмма последовательности (sequence diagram) --- это наглядное
представление совокупности разных элементов модели системы,
изображение того, как и в каком порядке они взаимодействуют.\par
Такие диаграммы подробно описывают, как выполняются разные операции.
При этом они показывают временной порядок или хронологию:
то, когда, как и в какой очереди передаются сообщения.\par
Диаграммы удобно использовать при проектировании или проверке архитектуры,
логики системы или интерфейса.\rref{fig:sequence}.

\begin{image}
	\includegrph[scale=0.27]{sec}
	\caption{Диаграмма последовательности}
	\label{fig:sequence}
\end{image}


\subsection{Выбор и обоснование архитектурного
	паттерна проектирования кода приложения}

Для разработки системы, которая управляет продуктами, их характеристиками и рекомендациями, выбран паттерн \textbf{<<Фабрика>>} (Factory Method).\par
Паттерн <<Фабрика>> используется для создания объектов без указания точного типа создаваемого объекта. Это позволяет упростить процесс создания различных типов объектов в системе, таких как \textbf{Product}, \textbf{Rate}, \textbf{Attribute}, \textbf{Vector} и другие, обеспечивая гибкость и модульность системы.\par
Каждый объект (например, продукт, оценка или атрибут) создается с помощью отдельной фабрики, которая инкапсулирует детали создания объектов и позволяет добавлять новые типы объектов без изменения существующего кода.\par
Этот паттерн идеален для систем, где необходимо часто создавать разнообразные объекты с различными параметрами и характеристиками, такие как описание продукта, его атрибуты, леммы и другие параметры.\par
Использование паттерна <<Фабрика>> также позволяет эффективно управлять зависимостями и упрощает тестирование, так как объекты создаются через фабричные методы, а не напрямую, что облегчает внедрение мок-объектов или подделок для тестов.\par
Таким образом, паттерн <<Фабрика>> подходит для разработки системы, обеспечивая гибкость в создании объектов, упрощая добавление новых типов объектов и повышая масштабируемость и модульность всей системы.

\clearpage  % XXX: Fine-tuning

\section{Путь пользователя}

Путь пользователя --- это общий алгоритм работы с продуктом. Так
называемый User Flow или путь пользователя, это последовательный
список действий или экранов, по которым может переходить
пользователь в процессе взаимодействия с продуктом.
Как пользователь будет взаимодействовать с продуктом
продемонстрированно на рисунке~\ref{fig:user:flow}.

\begin{image}
	\includegrph[scale= 0.9]{user_road.drawio.png}
	\caption{Путь пользователя в разработке}
	\label{fig:user:flow}
\end{image}

В результате построения такой диаграммы можно четко увидеть,
как взаимодействуют разные компоненты системы,
начиная от запуска программы до получения итогового результата.
Этот путь помогает лучше понять логику работы программы
и предоставляет ясное представление о том,
как пользователь будет взаимодействовать с системой на каждом этапе.

\section{Проектирование интерфейсов}

\subsection{Описание используемых технологий и их обоснование}

\subsubsection{Язык программирования}

Kotlin c html являются базовыми языками для разработки мобильных приложений на android.
Разметка может быть выполнена на любом из данных языков.

\subsubsection{Библиотека Compose}

Jetpack Compose — это современная декларативная библиотека для создания пользовательских интерфейсов на языке
программирования Kotlin в экосистеме Android.
Она разрабатывается Google как часть набора инструментов Jetpack и представляет собой более упрощенный и
интуитивно понятный подход к разработке UI по сравнению с классическими XML-макетами.
Compose позволяет описывать интерфейсы декларативно, что облегчает их создание, тестирование и модификацию.
\subsection{Описание интерфейса}

Давайте рассмотрим меж-экранные взаимодействия в мобильном клиент-серверном приложении.
\begin{itemize}
	\item Экран авторизации:
	\begin{itemize}
		\item Пользователь вводит свои учетные данные (логин и пароль).
		\item Приложение отправляет запрос на сервер
		для проверки подлинности.
		\item В случае успешной аутентификации пользователь
		перенаправляется на главный экран приложения.
		\item Пользователь может перейти на экран регистрации.
		\item Учетные данные могут быть сохранены в приложении.
	\end{itemize}
	\item Экран регистрации:
	\begin{itemize}
		\item Пользователь вводит свои учетные данные (логин и пароль).
		\item Приложение отправляет запрос на сервер
		для проверки.
		\item В случае успешной регистрации пользователь
		перенаправляется на экран авторизации.
	\end{itemize}
	\item Главный экран:
	\begin{itemize}
		\item Пользователь может выбрать фрагмент поиска фильма.
		\item Пользователь может выбрать фрагмент личных рекомендаций.
	\end{itemize}
	\item Фрагмент поиска фильма:
	\begin{itemize}
		\item Пользователь может найти фильм по его названию.
		\item Пользователь может перейти в экран фильма.
	\end{itemize}
	\item Фрагмент личных рекомендаций:
	\begin{itemize}
		\item Пользователь получает личные рекомендации на основе выставленных лайков.
		\item Пользователь может перейти в экран фильма.
	\end{itemize}
	\item Экран фильма:
	\begin{itemize}
		\item Пользователь может прочитать данные о фильме.
		\item Пользователь может оценить фильм как понравившийся и как не понравившийся.
	\end{itemize}
\end{itemize}

В каждом из этих сценариев меж-экранные взаимодействия включают отправку запросов на сервер для получения или
обновления данных, а также отображение результата операции на мобильном устройстве пользователя.
Это обеспечивает плавное и эффективное взаимодействие между клиентским и серверным компонентами приложения.


\subsubsection{Создание экранов приложения}

В современном мире мобильных и веб-приложений, пользовательский интерфейс (UI) играет ключевую роль в восприятии и
успешности продукта.
Создание экранов приложения — это не просто процесс проектирования и программирования отдельных элементов интерфейса.
Это искусство, требующее глубокого понимания потребностей пользователя, принципов дизайна и технических возможностей
платформы.

Экран авторизации: важный компонент для обеспечения безопасности и персонализированного доступа к системе,
разработанный в рамках данной курсовой работы.

\begin{image}
	\includegrph[scale=0.28]{login}
	\caption{Экран авторизации}
	\label{engineering:login}
\end{image}

\clearpage

Экран регистрации: интерфейс для создания новых учетных записей, обеспечивающий удобный и безопасный процесс
регистрации пользователей.

\begin{image}
	\includegrph[scale=0.28]{signup}
	\caption{Экран регистрации}
	\label{engineering:signup}
\end{image}
Экран поиска: удобный инструмент для быстрого и эффективного нахождения нужной информации в системе.
\begin{image}
	\includegrph[scale=0.28]{find}
	\caption{Экран поска}
	\label{engineering:find}
\end{image}
\clearpage
Экран оценки: простой и наглядный способ выразить свое мнение о фильме, выбрав либо лайк, либо дизлайк.

\begin{image}
	\includegrph[scale=0.28]{film}
	\caption{Экран фильма}
	\label{engineering:film}
\end{image}

Экран рекомендаций: персонализированные рекомендации фильмов, основанные на оценках и предпочтениях пользователей,
чтобы помочь вам открыть новые кинематографические шедевры.
\begin{image}
	\includegrph[scale=0.40]{recom}
	\caption{Экран рекоминдаций}
	\label{engineering:recom}
\end{image}

