\chapter{Исследовательский раздел}

\section{Моделирование бизнес-процессов}

\subsection{Входные и выходные данные}
Входные данные:

\begin{itemize}
	\item api и сайт Кинопоиска;
	\item Оценки пользователей о фильме.
\end{itemize}

Выходные данные:

\begin{itemize}
	\item Мобильный клиент для просмотра и оценки фильмов;
	\item Индивидуальные рекомендации для пользователей на основе их оценки и описания фильма.
\end{itemize}

\subsection{Модель БП в вариации TO-BE}

Модель бизнес-процессов системы рекомендации в формате TO-BE демонстрирует целевую архитектуру процесса \rref{fig:idef0}.
\begin{image}
	\includegrph[scale=0.2]{01_A-0}
	\caption{Контекстная диаграмма}
	\label{fig:idef0}
\end{image}

Процесс работы рекомендательной системы декомпозируется в 4 процесса \rref{fig:idef0:a0}.

\begin{image}
	\includegrph[scale=0.25]{02_A0}
	\caption{Декомпозиция контекстной диаграммы}
	\label{fig:idef0:a0}
\end{image}


Целевая модель TO-BE определяет четкую последовательность обработки данных,
что повышает скорость разработки
и упрощает дальнейшую модификацию или масштабирование системы.

\section{Границы проекта}

Границы автоматизируемых бизнес-процессов включают:

\begin{itemize}
	\item Сбор информации о фильмах;
	\item Выдача информации о фильмах;
	\item Работа с данными клиентов;
	\item Создание рекомендаций.
\end{itemize}

Границы автоматизации не включают:

\begin{itemize}
	\item Изменение данных о фильмах;
	\item Взаимодействие клиентов;
	\item Авторизация через внешние сервисы;
	\item Рассылка новостей;
	\item Предоставление внешнего api.
\end{itemize}

\section{Анализ систем-аналогов}

Большая часть рекомендательных систем уже встроенно в стриминговые сервисы.
Рассмотрим как именно они работают.
\begin{itemize}
	\item Netflix; \par
	Netflix предлагает одну из самых продвинутых систем рекомендаций для фильмов и сериалов.
	Они используют алгоритмы машинного обучения, такие как коллаборативная фильтрация и контентные алгоритмы, чтобы
	адаптировать рекомендации к предпочтениям каждого пользователя.

	\item Amazon Prime Video; \par
	Amazon также использует алгоритмы машинного обучения для предлагания рекомендаций фильмов и телешоу,
	основанных на предпочтениях и поведении пользователей.

	\item IMDb; \par
	IMDb предоставляет оценки и рецензии пользователей, которые могут быть использованы для рекомендаций.
	Они также используют информацию о предпочтениях пользователя и о схожести фильмов для предложения контента.

	\item Okko (ОККО); \par
	Это российская платформа для просмотра фильмов и сериалов онлайн, которая также имеет свою систему рекомендаций.
	Они используют алгоритмы машинного обучения для адаптации предложений под интересы пользователей.

	\item ivi (Иви);\par
	Еще одна популярная российская платформа для просмотра фильмов и сериалов с собственной системой рекомендаций.
	Они также используют алгоритмы машинного обучения для предложения контента пользователям.

	\item КиноПоиск;\par
	Это российский интернет-портал о кино, который предлагает рецензии, рейтинги и рекомендации для фильмов и сериалов.
	Они также используют данные о предпочтениях пользователей и оценках для предложения контента.

\end{itemize}

\paragraph{Netflix}\par
Netflix предлагает одну из самых продвинутых систем рекомендаций для фильмов и сериалов.
Они используют алгоритмы машинного обучения, такие как коллаборативная фильтрация и контентные алгоритмы, чтобы
адаптировать рекомендации к предпочтениям каждого пользователя.
\begin{itemize}
	\item Коллаборативная фильтрация:
	Netflix использует информацию о рейтингах, просмотрах и предпочтениях пользователей для поиска схожих пользователей
	и рекомендации контента, который понравился этим пользователям, но еще не просмотрен другими.
	\item Контентные алгоритмы:
	Алгоритмы анализируют характеристики контента (например, жанр, актеры, режиссеры) и историю просмотров пользователя,
	чтобы предложить фильмы и сериалы, которые могут быть интересны на основе их предпочтений.
	\item Алгоритмы, основанные на поведении пользователя:
	Netflix анализирует поведение пользователя на платформе, такие как остановки, перемотки, добавления в список
	просмотра и т. д., чтобы понять их интересы и предпочтения.
\end{itemize}

\paragraph{Amazon Prime Video}

Amazon также использует алгоритмы машинного обучения для предлагания рекомендаций фильмов и телешоу,
основанных на предпочтениях и поведении пользователей.

\begin{itemize}
	\item Коллаборативная фильтрация:
	Amazon использует данные о рейтингах и просмотрах, чтобы предложить контент, который был популярен среди
	пользователей с похожими вкусами.
	\item Контентные алгоритмы:
	Подобно Netflix, Amazon анализирует характеристики контента и предпочтения пользователей для рекомендации фильмов
	и сериалов.
	\item Алгоритмы, основанные на поведении пользователя:
	Amazon анализирует действия пользователя на платформе для предложения контента, который может быть интересен.
\end{itemize}

\paragraph{Hulu, IMDb, Okko, ivi, КиноПоиск}

Как правило, эти платформы также используют комбинацию коллаборативной фильтрации, контентных алгоритмов и алгоритмов,
основанных на поведении пользователя, для рекомендации контента, который наиболее подходит для каждого пользователя.

\begin{itemize}
	\item IMDb; \par
	IMDb предоставляет оценки и рецензии пользователей, которые могут быть использованы для рекомендаций.
	Они также используют информацию о предпочтениях пользователя и о схожести фильмов для предложения контента.

	\item Okko (ОККО); \par
	Это российская платформа для просмотра фильмов и сериалов онлайн, которая также имеет свою систему рекомендаций.
	Они используют алгоритмы машинного обучения для адаптации предложений под интересы пользователей.

	\item ivi (Иви);\par
	Еще одна популярная российская платформа для просмотра фильмов и сериалов с собственной системой рекомендаций.
	Они также используют алгоритмы машинного обучения для предложения контента пользователям.

	\item КиноПоиск.\par
	Это российский интернет-портал о кино, который предлагает рецензии, рейтинги и рекомендации для фильмов и сериалов.
	Они также используют данные о предпочтениях пользователей и оценках для предложения контента.
\end{itemize}
\clearpage

\clearpage  % XXX: Fine-tuning

\section{Техническое задание}

\subsection{Функциональные, нефункциональные и пользовательские требования}

\subsubsection{Функциональные требования}

Эти требования описывают, что система должна делать. Они определяют конкретные функции,
которые система должна выполнять для решения задач пользователя.

Перечислим их:

\begin{itemize}
	\item Система должна принимать на вход данные о фильмах ;
	\item Система должна принимать на вход данные о пользователях (история просмотров, оценки, предпочтения);
	\item Система должна уметь считывать конфигурационный файл с параметрами алгоритма рекомендаций;
	\item Система должна обрабатывать предварительную настройку параметров, заданных в конфигурационном файле, до формирования рекомендаций;
	\item Система должна вести журнал (логирование) всех выполненных операций для последующего анализа;
	\item Система должна формировать итоговый список рекомендованных фильмов;
\end{itemize}

\subsubsection{Пользовательские требования}

Эти требования формируются с точки зрения пользователя и описывают задачи,
которые пользователь должен выполнить с помощью системы.
Они менее технически сложны и более ориентированы на удобство взаимодействия.

Перечислим их:

\begin{itemize}
	\item Система должна предоставлять интерфейс для авторизации;
	\item Система должна предоставлять интерфейс для просмотра данных и оценки фильма;
	\item Интерфейс системы должен быть реализован в виде мобильного приложения, понятного для пользователя;
	\item Система должна предоставлять отчёт о рекомендациях по запросу пользователя.
\end{itemize}

\subsubsection{Нефункциональные требования}

Эти требования касаются характеристик системы, определяющих,
как она выполняет свои функции, но не связанных непосредственно с её поведением.
Они включают такие аспекты, как производительность, надёжность, безопасность,
удобство использования и масштабируемость.

Перечислим их:

\begin{itemize}
	\item Время формирования рекомендаций должно быть разумным и не превышать нескольких минут для типового набора данных;
	\item Система должна быть реализована на языке программирования (например, Python 3.9);
	\item Система не должна изменять исходные данные о пользователях или фильмах;
\end{itemize}

\subsection{Требования к программному обеспечению}

\subsubsection{Состав ПО}

Сиcтема состоит из следующих подсистем:
\begin{itemize}
	\item База данных;
	\item Модуль рекомендации;
	\item API;
	\item Парсер;
	\item Мобильное приложение.
\end{itemize}

База данных -- xранит информацию о: фильмах, оценках и пользователях.
Модуль рекомендации -- предоставляет личные рекомендации пользователям на основе их оценок.
API -- выполняет функцию для связки всех подсистем.
Мобильное приложение -- предоставляет пользователю интерфейс общения с системой.
Парсер -- собирает новые фильмы и информацию о них.

\subsubsection{Выбор ПО}

Язык программирования: Python (версии 3.10 и выше).

Библиотеки Python:

\begin{itemize}
	\item scikit-learn;
	\item fastapi;
	\item catboost;
	\item nltk;
	\item pymystem3;
	\item navec.
\end{itemize}

\subsubsection{Разрабатываемое ПО}


Реализация алгоритма личной рекомендации фильмов, основанный на текстовом описании фильма и оценках пользователя.
С мобильным приложением в виде интерфейса пользователя.

\subsubsection{Покупные средства}

Использование библиотек с открытым исходным кодом.

\subsection{Требования к техническому обеспечению}

\subsection*{Технические средства.}
Персональные сервера на ОС Linux.

\subsection*{Функциональные характеристики}

\subsubsection*{Общие возможности системы}
\begin{itemize}
	\item Обеспечение стабильной работы при нагрузке до 1\,000 одновременно активных пользователей.
	\item Обработка до 10 запросов в секунду при средней задержке не более 200 миллисекунд.
	\item Поддержка масштабирования до 100\,000 зарегистрированных пользователей.
	\item Интеграция с внешними API для получения данных с частотой обновления до 1 запроса в 10 секунд.
\end{itemize}

\subsubsection*{Требования к серверной части}
\begin{itemize}
	\item Поддержка REST API с пропускной способностью не менее 500 Мбит/с.
	\item Обеспечение устойчивости к сбоям с доступностью 99.9\% (время простоя не более 8 часов в год).
	\item Шифрование данных с использованием алгоритма AES-256.
	\item Возможность обработки до 1\,000 транзакций в минуту.
\end{itemize}

\subsubsection*{Требования к базе данных}
\begin{itemize}
	\item Реляционная база данных (например, PostgreSQL) с максимальным размером таблиц до 10 ТБ.
	\item Среднее время выполнения сложных запросов: не более 500 миллисекунд.
	\item Хранение информации о 100\,000 фильмов, включая:
	\begin{itemize}
		\item Название, описание, жанры, рейтинг, постеры.
		\item Данные о пользователях: профили, история просмотров, предпочтения.
		\item Логи взаимодействий объёмом до 100 ГБ в месяц.
	\end{itemize}
	\item Репликация данных в режиме реального времени с задержкой не более 1 секунды.
\end{itemize}

\subsubsection*{Требования к клиентской части}
\begin{itemize}
	\item Совместимость с устройствами под управлением Android версии 8.0+ .
	\item Максимальный размер установочного файла: 50 МБ.
	\item Потребление оперативной памяти не более 300 МБ при активной работе.
	\item Уведомления с задержкой доставки не более 5 секунд.
	\item Возможность работы в режиме офлайн с хранением до 500 записей в локальной базе данных.
\end{itemize}

\subsubsection*{Требования к оборудованию}
\begin{itemize}
	\item \textbf{Серверная часть:}
	\begin{itemize}
		\item Процессор: минимум 8 ядер с частотой 2.5 ГГц.
		\item Оперативная память: минимум 32 ГБ.
		\item Хранилище: SSD от 1 ТБ с возможностью масштабирования до 10 ТБ.
		\item Сетевое соединение: симметричный канал от 1 Гбит/с.
	\end{itemize}
	\item \textbf{Клиентская часть:}
	\begin{itemize}
		\item Минимальные устройства: смартфоны с 2 ГБ оперативной памяти и 16 ГБ свободного места.
		\item Экран: поддержка разрешения 720p и выше.
	\end{itemize}
\end{itemize}

\subsubsection*{Требования к производительности системы}
\begin{itemize}
	\item Время отклика пользовательского интерфейса: не более 100 миллисекунд.
	\item Средняя загрузка процессора сервера: не более 70\% при пиковой нагрузке.
	\item Возможность восстановления системы после сбоя в течение 10 минут.
	\item Среднее время обновления данных в базе: не более 1 секунды.
\end{itemize}

%Функциональные характеристики.
%Обеспечение устойчивой работы при больших объемах данных.

\subsection{Требования к информационному обеспечению}

\subsubsection{Состав и структура данных}

Рекомендательная система фильмов оперирует следующими типами данных:
\begin{itemize}
	\item Оценки и отзывы, отражающие мнение пользователей о просмотренных фильмах;
	\item Метаинформация о фильмах (жанр, год выпуска, актёрский состав, страна производства);
	\item Конфигурационные файлы, содержащие параметры рекомендационного алгоритма (например, весовые коэффициенты, гиперпараметры моделей).
\end{itemize}

Данные структурированы по логическим модулям, чтобы обеспечить удобный доступ,
систематизацию и обработку необходимых для формирования персональных рекомендаций сведений.

\subsubsection{Организация данных}

Информация хранится в текстовых форматах (например, JSON или CSV), доступных для чтения и редактирования.
Файлы могут быть размещены на локальном устройстве или в облачном хранилище.
Конфигурационные файлы содержат настройки и параметры алгоритма,
обеспечивая корректную работу всей системы.

\subsubsection{Информационный обмен}

Взаимодействие между модулями системы (сбор данных, предобработка, обучение модели, формирование рекомендаций)
должно быть организовано так, чтобы обеспечить непрерывный поток информации и правильную
последовательность выполнения операций. Это необходимо для корректной интеграции данных о пользователях,
фильмах и результатах работы рекомендационного алгоритма.

\subsubsection{Информационная совместимость}

Система должна обеспечивать совместимость со стандартными форматами данных,
используемыми для описания фильмов (например, данные из внешних API-источников),
а также с форматами пользовательских данных, экспортируемыми из различных
сервисов или приложений. Это необходимо для расширения функциональности,
масштабируемости и интеграции со сторонними источниками данных.

\subsubsection{Классификаторы и справочники}

Для систематизации информации о фильмах и аудитории необходимо использование
справочников и классификаторов (например, справочник жанров, классификатор возрастных групп,
справочник стран производства). Это обеспечивает более точную фильтрацию и персонализацию рекомендаций.

\subsubsection{Системы управления базами данных}

Использование реляционных или нереляционных СУБД может потребоваться
при усложнении системы или увеличении объёмов данных.


\subsubsection{Контроль и восстановление данных}

Система должна вести логи работы, фиксируя каждую операцию, а также возникающие ошибки.

\subsection{Требования к математическому обеспечению}

Для реализации математического обеспечения рекомендационной системы фильмов
требуются следующие алгоритмы и методы:

\begin{itemize}
	\item Алгоритмы машинного обучения для построения моделей рекомендаций
	(например, коллаборативная фильтрация, факторизация матриц,
	нейронные сети или методы обучения с подсказками);
	\item Методы обработки и анализа пользовательских данных, включая нормализацию,
	фильтрацию шумов, выделение признаков и агрегацию поведенческих метрик;
	\item Алгоритмы обработки естественного языка (NLP) для анализа текстовой информации,
	такой как описания фильмов или отзывы пользователей
	(токенизация, лемматизация, определение ключевых слов и фраз);
	\item Методы оценки качества рекомендаций, включая расчёт метрик точности (Precision, Recall, F1),
	а также использование A/B-тестирования и других экспериментов
	для валидирования эффективности рекомендационного алгоритма.
\end{itemize}

\subsection{Требования к документации на программное решение}

\subsubsection{Перечень подлежащих разработке документов}

\begin{itemize}
	\item Техническое задание (ТЗ);
	\item Проектная документация (проект, спецификации);
	\item Программа и методика испытаний;
	\item Руководства пользователя и администраторов;
	\item Отчеты о внедрении и эксплуатации системы.
\end{itemize}

\subsubsection{Вид представления и количество документов}

\textbf{Формат}:

Все документы предоставляются в электронном формате (PDF),
удобном для хранения и печати,
а также в текстовом формате (Word или аналог) для возможных правок.

\textbf{Количество экземпляров}:

По одному электронному экземпляру каждой документации
для заказчика и для внутреннего хранения у разработчика.

При необходимости печатные версии, по одному экземпляру Технического задания,
Руководства пользователя и Акта приемки для заказчика.

\subsection{Требования к надежности программного решения}

АС должна обеспечивать корректное выполнение рекомендации
при каждом запуске по запросу пользователя.
Вероятность аварийного завершения работы системы должна быть
не более одного случая на 1000 запусков.

При возникновении ошибок или сбоев система должна отправлять код ошибки
с выдачей соответствующего сообщения об ошибке.
Все ошибки логируются для последующего анализа и устранения.

\subsection{Требования к безопасности программного решения}

Данные авторизации пользователей должны быть зашифрованны.

\section{Программа и методика испытаний}

\subsection{Объект испытаний}
\subsubsection{Наименование системы}

Мобильное приложение для рекомендаций фильмов

\subsubsection{Область применения системы}

Программный продукт представляет собой рекомендательную систему для фильмов, основанную на анализе описаний с использованием дерева решений. Основная цель системы — предложить пользователю персонализированные рекомендации фильмов на основе их интересов и предпочтений. Android-приложение взаимодействует с серверной частью через Rest API, которая обрабатывает запросы, используя данные, хранящиеся в базе данных Postgres.

Основные функции системы включают:

\begin{itemize}
	\item Рекомендация фильмов на основе анализа описаний и предпочтений пользователя с помощью дерева решений.
	\item Взаимодействие с сервером для получения и отправки данных через Rest API.
	\item Хранение и управление данными о фильмах и предпочтениях пользователей в базе данных Postgres.
	\item Предоставление интерфейса для поиска и получения рекомендаций в мобильном приложении.
\end{itemize}

Области применения:

\begin{itemize}
	\item Использование в мобильных приложениях для рекомендаций фильмов, чтобы улучшить пользовательский опыт и предложить фильмы, соответствующие интересам пользователей.
	\item Интеграция в существующие онлайн-платформы и приложения, занимающиеся предоставлением медиа-контента.
	\item Применение для исследовательских целей в области машинного обучения и рекомендательных систем для улучшения точности рекомендаций.
\end{itemize}

\subsubsection{Условное обозначение системы}

Условное обозначение системы — РСФ (Рекомендательная Система Фильмов).

\subsection{Цель испытаний}

Целью проводимых по настоящей программе и методике испытаний РСФ (Рекомендательной Системы Фильмов) является определение функциональной работоспособности системы на этапе проведения испытаний.

Программа испытаний должна удостоверить работоспособность РСФ в соответствии с её функциональным предназначением — предоставление пользователю точных рекомендаций фильмов на основе анализа описаний с использованием дерева решений, корректной работы с базой данных Postgres, а также надёжного взаимодействия через Rest API с мобильным Android-приложением.

\subsection{Общие положения}

\subsubsection{Перечень руководящих документов, на основании которых проводятся испытания}

Приёмочные испытания РСФ (Рекомендательной Системы Фильмов) проводятся на основании следующих документов:

\begin{itemize}
	\item Утверждённое Техническое задание на разработку РСФ;
	\item Настоящая Программа и методика приёмочных испытаний;
\end{itemize}

\subsubsection{Место и продолжительность испытаний}

Место проведения испытаний — площадка Заказчика.
Продолжительность испытаний устанавливается Приказом Заказчика о составе приёмочной комиссии и проведении приёмочных испытаний.

\subsubsection{Организации, участвующие в испытаниях}

В приёмочных испытаниях участвуют представители следующих организаций:

\begin{itemize}
	\item ООО \("\)Кинопоиск\("\) (Заказчик);
	\item Шестаокв Ярослев Евгеньевич (Исполнитель).
\end{itemize}

Конкретный перечень лиц, ответственных за проведение испытаний системы, определяется Заказчиком.
\subsubsection{Перечень предъявляемых на испытания документов}

Для проведения испытаний Исполнителем предъявляются следующие документы:

\begin{itemize}
	\item Техническое задание на создание РСФ;
	\item Технический проект РСФ;
	\item Программа и методика испытаний на РСФ;
\end{itemize}
\subsection{Объём испытаний}

\subsubsection{Перечень этапов испытаний и проверок}

В процессе проведения приёмочных испытаний должны быть протестированы следующие подсистемы РСФ (Рекомендательной Системы Фильмов):

\begin{itemize}
	\item Подсистема обработки пользовательского ввода в Android-приложении;
	\item Подсистема анализа данных фильмов с использованием дерева решений;
	\item Подсистема взаимодействия с базой данных Postgres;
	\item Подсистема Rest API для обработки запросов;
	\item Подсистема генерации рекомендаций.
\end{itemize}

Все подсистемы испытываются одновременно на корректность взаимодействия, влияние одной подсистемы на другие, то есть испытания проводятся комплексно.

Приёмочные испытания включают проверку:

\begin{itemize}
	\item полноты и качества реализации функций, указанных в ТЗ;
	\item выполнения каждого требования, относящегося к интерфейсу Android-приложения;
	\item работы пользователей в диалоговом режиме;
	\item полноты действий, доступных пользователю, и их достаточности для функционирования системы;
	\item простоты использования приложения, возможности работы пользователей без специальной подготовки;
	\item реакции системы на ошибки пользователя;
	\item практической выполнимости рекомендованных процедур.
\end{itemize}

\subsubsection{Испытания подсистемы обработки пользовательского ввода}

Испытания подсистемы обработки пользовательского ввода направлены на проверку доступности всех функций приложения, удобства использования интерфейса, корректного ввода данных и обработки ошибок. Также проверяется устойчивость приложения при сбоях и стабильность работы при различных сценариях пользовательского ввода.

\subsubsection{Испытания подсистемы анализа данных фильмов}

Тестируется корректность работы дерева решений для генерации рекомендаций на основе описаний фильмов и пользовательских предпочтений. Проверяется точность модели, реакция на некорректные или неполные данные, а также производительность при больших объёмах данных.

\subsubsection{Испытания подсистемы взаимодействия с базой данных Postgres}

Проверяется правильность записи и чтения данных из базы данных Postgres, включая корректную обработку запросов, отказоустойчивость, а также производительность при большом количестве запросов и данных.

\subsubsection{Испытания подсистемы Rest API}

Основная цель — убедиться в правильности обработки запросов через Rest API, включая генерацию корректных ответов, обработку ошибок, а также устойчивость при высокой нагрузке и сбоях. Тестируется производительность и надёжность обмена данными между клиентом и сервером.

\subsubsection{Испытания подсистемы генерации рекомендаций}

Тестируется полнота и точность генерируемых рекомендаций, проверяется их соответствие интересам пользователя. Также тестируется производительность подсистемы и её способность обрабатывать запросы при большом количестве пользователей.

\subsection{Методика проведения испытаний}

\begin{longtable}{|c|p{7.5cm}|p{7.5cm}|}
	\caption{\leftline{Методика проведения испытаний}} \label{table:test} \\
	\hline
	\textbf{\No} & \textbf{Действие} & \textbf{Результат} \\
	\hline
	\endfirsthead
	\conttable{table:test} \\
	\hline
	\textbf{\No} & \textbf{Действие} & \textbf{Результат} \\
	\hline
	\endhead

	\textbf{1}
	& \multicolumn{2}{|l|}{\textbf{Сценарий <<Тестирование обработки пользовательского ввода>>}} \\ \hline
	1.1
	& Ввести некорректные данные (например, неверный запрос в Android-приложении) и попытаться получить рекомендации.
	& Отображается сообщение об ошибке, приложение не завершает работу аварийно. \\ \hline

	1.2
	& Ввести корректные данные, запросить рекомендации по фильму.
	& Запрос успешно обработан, отображаются результаты рекомендаций. \\ \hline

	1.3
	& Проверить лог действий в приложении.
	& Лог корректно отображает все этапы обработки запроса. \\ \hline

	\textbf{2}
	& \multicolumn{2}{|l|}{\textbf{Сценарий <<Тестирование анализа данных фильмов>>}} \\ \hline
	2.1
	& Передать описание фильма с неполными данными в подсистему анализа.
	& Программа корректно обрабатывает данные, генерирует рекомендации на основе доступной информации. \\ \hline

	2.2
	& Передать большое количество описаний фильмов и запросить рекомендации.
	& Подсистема успешно обрабатывает данные, время выполнения соответствует ожидаемым параметрам. \\ \hline

	\textbf{3}
	& \multicolumn{2}{|l|}{\textbf{Сценарий <<Тестирование взаимодействия с базой данных Postgres>>}} \\ \hline
	3.1
	& Выполнить запрос на получение данных о фильмах из базы Postgres.
	& Данные успешно получены, отображаются корректно в приложении. \\ \hline

	3.2
	& Выполнить запрос с большими объемами данных (1000+ фильмов).
	& Запрос выполнен без ошибок, производительность соответствует заявленным требованиям. \\ \hline

	\textbf{4}
	& \multicolumn{2}{|l|}{\textbf{Сценарий <<Тестирование Rest API>>}} \\ \hline
	4.1
	& Выполнить запрос на получение рекомендаций через Rest API.
	& API возвращает корректный результат, ошибки отсутствуют. \\ \hline

	4.2
	& Выполнить тестирование на высокую нагрузку (100+ параллельных запросов).
	& Rest API выдерживает нагрузку, время ответа в пределах допустимых значений. \\ \hline

	\textbf{5}
	& \multicolumn{2}{|l|}{\textbf{Сценарий <<Тестирование подсистемы генерации рекомендаций>>}} \\ \hline
	5.1
	& Запросить рекомендации на основе описания фильма с различными жанрами и параметрами.
	& Система корректно генерирует рекомендации в соответствии с предпочтениями пользователя. \\ \hline

	5.2
	& Проверить корректность рекомендаций на основе пользовательских оценок.
	& Рекомендации соответствуют предпочтениям и параметрам, указанным пользователем. \\ \hline
\end{longtable}

\subsection{Требования по испытаниям программных средств}

Испытания программных средств РСФ проводятся в процессе функционального тестирования системы и нагрузочного тестирования. Других требований по испытаниям программных средств РСФ не предъявляется.

\subsection{Перечень работ, проводимых после завершения испытаний}

По результатам испытаний составляется заключение о соответствии РСФ требованиям ТЗ и возможности оформления акта сдачи в эксплуатацию. При необходимости проводится доработка системы и документации.

\subsection{Условия и порядок проведения испытаний}

Испытания РСФ проводятся на оборудовании Заказчика с использованием Android-устройств и серверов для хостинга базы данных и Rest API. Во время испытаний проводится полное функциональное тестирование в соответствии с требованиями ТЗ.

\subsection{Материально-техническое обеспечение испытаний}

Испытания проводятся на оборудовании Заказчика с следующей минимальной конфигурацией:

\begin{itemize}
	\item Android-устройство для тестирования приложения;
	\item Сервер с установленной базой данных Postgres;
	\item Сервер для Rest API;
	\item Операционная система: Linux.
\end{itemize}

\subsection{Метрологическое обеспечение испытаний}

Испытания РСФ не требуют использования специализированного измерительного оборудования.

\subsection{Отчётность}

Результаты испытаний фиксируются в протоколах, содержащих информацию о выполненных тестах, используемых технических средствах, методиках проведения тестирования и оценке результатов. В конце испытаний оформляется акт сдачи РСФ в эксплуатацию.

\clearpage


