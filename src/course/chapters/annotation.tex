%\begin{abstract}

\chapter*{Аннотация}
%\addcontentsline{toc}{chapter}{Аннотация}

Выпускная квалификационная работа посвящена разработке
программы-конвертера DRC правил,
предназначенной для автоматизации процесса перевода проектных правил
между различными CAD-системами.
В условиях стремительного развития электронной промышленности
и увеличения сложности проектируемых устройств возникает необходимость
в инструментах, способствующих повышению эффективности разработки
и минимизации ошибок.

Работа начинается с анализа существующих методов
и инструментов проверки проектных правил,
а также с исследования форматов представления DRC правил
в популярных CAD-системах.
На основе проведенного анализа разрабатывается архитектура
программы-конвертера, описываются алгоритмы преобразования правил
и осуществляется реализация программного обеспечения.

В ходе тестирования программа демонстрирует высокую точность
и эффективность в преобразовании правил,
что подтверждает её способность существенно сократить время,
необходимое для подготовки проектной документации.
Результаты работы могут быть использованы как в академических,
так и в промышленных целях,
способствуя интеграции различных инструментов проектирования
и повышению качества разработки электронных устройств.

Ключевые слова:
DRC правила, CAD-системы, программа-конвертер,
автоматизация проектирования, электронные устройства.

Расчетно-пояснительная записка
состоит из \pageref{lastpage} листов,
содержит
\totalfigures рисунков
, \totaltables таблиц
, \total{appendixcount} приложение.

