\chapter*{Заключение}
\addcontentsline{toc}{chapter}{Заключение}
В ходе выполнения данной курсовой работы была разработана экспериментальная рекомендательная система фильмов,
предназначенная для автоматизации процесса формирования персональных рекомендаций.
Изучение актуальности темы подтвердило востребованность решений, которые способны упростить выбор контента
пользователями, что и стало основной целью данной работы.

Реализация системы позволила решить поставленные задачи, включающие анализ существующих алгоритмов рекомендаций,
исследование применяемых методов анализа пользовательских предпочтений и разработку архитектуры программного обеспечения.
Созданный прототип был протестирован на различных сценариях и наборах данных, что продемонстрировало его высокую точность
и эффективность в формировании персонализированных рекомендаций.

Результаты тестирования подтвердили, что использование разработанной системы позволяет значительно сократить время
на подбор интересного контента, повысить релевантность получаемых рекомендаций, а также улучшить пользовательский опыт.

Таким образом, разработанное программное решение представляет собой значительный вклад в автоматизацию и улучшение
качества пользовательских рекомендаций в сфере онлайн-просмотра фильмов.
Данное решение может быть полезно как для проведения научных исследований в области рекомендации контента, так и для
практического применения в индустрии, включая онлайн-кинотеатры, стриминговые платформы и мобильные приложения.

В дальнейшем возможна доработка системы, включая расширение функциональности, поддержку новых источников данных
и совершенствование алгоритмов машинного обучения для ещё более точного учёта индивидуальных предпочтений.
Это обеспечит ещё большую универсальность и применимость инструмента в различных областях, связанных с персонализацией контента.

В заключение, данная работа подтверждает важность интеграции современных алгоритмических и аналитических подходов в процесс
разработки рекомендательных систем, что в конечном итоге способствует повышению качества, удобства и персонализации пользовательского опыта.

