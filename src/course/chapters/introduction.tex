\chapter*{Введение}
\addcontentsline{toc}{chapter}{Введение}
\textit{Актуальность.}
С развитием цифровых технологий и ростом объёмов информации системы рекомендаций становятся важным инструментом,
помогающим пользователям ориентироваться в огромном количестве доступных данных.
Одной из наиболее востребованных сфер применения таких систем являются платформы для просмотра фильмов.
Рекомендательные системы позволяют персонализировать опыт пользователя, повышая удовлетворённость и вовлечённость аудитории.
Значимость выбранной темы обусловлена необходимостью повышения качества рекомендаций, улучшения пользовательского опыта
и увеличения лояльности аудитории к сервису.
Эффективные алгоритмы рекомендаций способствуют оптимизации процесса выбора фильмов и обеспечивают конкурентные преимущества для платформы.

\underline{Цель работы:}
Разработка рекомендательной системы фильмов, которая автоматизирует и оптимизирует процесс предоставления
персонализированных рекомендаций на основе предпочтений пользователя.
Такое решение позволит сократить время поиска контента, повысить точность рекомендаций и улучшить пользовательский опыт.

\underline{Объект исследования:}
Процесс разработки программного обеспечения для автоматизации предоставления рекомендаций фильмов.

\underline{Предмет исследования:}
Методы машинного обучения, алгоритмы фильтрации данных и подходы к разработке рекомендательных систем, используемые для персонализации рекомендаций.

\underline{Значимость работы:}
Реализация данного проекта направлена на решение важной задачи улучшения пользовательского опыта за счёт внедрения
персонализированных рекомендаций.
Это соответствует современным тенденциям в сфере цифровых технологий и способствует повышению конкурентоспособности платформы для просмотра фильмов.

Для достижения поставленной цели необходимо решить следующие\underline{задачи}:

\begin{enumerate}
    \item Проанализировать существующие подходы и алгоритмы для построения рекомендательных систем;
    \item Исследовать способы хранения и обработки данных о предпочтениях пользователей и фильмах;
    \item Разработать архитектуру рекомендательной системы, включая модуль сбора данных, алгоритмы фильтрации и генерации рекомендаций;
    \item Реализовать программное обеспечение, включающее алгоритмы рекомендаций, и провести его тестирование на предмет точности и производительности;
    \item Оценить результаты работы системы и сформулировать рекомендации по её использованию и дальнейшему улучшению.
\end{enumerate}
Данная курсовая работа включает: введение, три главы, заключение,
список использованной литературы и приложение.

Введение предоставляет общий обзор выбранной темы, формулирует цель
и задачи исследования, а также обосновывает актуальность проблемы.
Этот глава объясняет,
почему данная проблема важна для современной инженерной практики
и какие преимущества дает ее решение.

Первая глава посвящена исследовательскому разделу,
в рамках которого проводится моделирование бизнес-процессов.
Описываются входные и выходные данные,
разрабатываются модели бизнес-процессов в вариации TO-BE,
определяется граница проекта.
Также проводится анализ существующих систем-аналогов
и формулируется техническое задание, включая функциональные, нефункциональные
и пользовательские требования.
Здесь же разрабатываются требования к программному, техническому,
информационному и математическому обеспечению.
Завершается глава описанием программы и методики испытаний,
включая объем, условия проведения и метрологическое обеспечение.

Вторая глава посвящена аналитическому разделу.
В этой части представлена диаграмма Ганта для планирования разработки,
обоснован выбор инструментов, таких как языки программирования,
библиотеки, фреймворки и системы управления версиями.
Также в главе разрабатывается архитектура программного решения,
включая описание компонентов, их масштабируемости и интеграции.
Здесь же представлена модель базы данных с логической моделью,
словарем данных и матрицей доступа.

Третья глава раскрывает проектная глава,
где рассматриваются детальные аспекты разработки программного решения.
Разрабатывается дерево функций,
описывается логика работы программного обеспечения
с использованием диаграмм прецедентов и последовательности.
Особое внимание уделяется проектированию интерфейсов,
описанию пути пользователя и выбору технологий для реализации интерфейса.
Также в этой главе проводится обоснование выбранного
архитектурного паттерна проектирования.

Заключение подводит итоги исследования,
формулирует основные выводы и предлагает перспективы дальнейших исследований.
Здесь акцентируется внимание на значимости полученных результатов
для повышения качества проектирования электронных устройств.

Список использованной литературы включает все источники,
использованные при выполнении курсовой работы, включая научные статьи,
книги, публикации и интернет-ресурсы.

Приложения содержат дополнительные материалы, такие как техническое задание,
разработанное в соответствии с ГОСТ 34.602-89,
и другие вспомогательные документы,
необходимые для более глубокого понимания результатов исследования.

