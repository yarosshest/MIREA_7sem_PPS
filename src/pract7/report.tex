\section{Диаграмма Ганта для разработки программного средства}

Диаграмма Ганта представляет собой визуальное отображение задач
и временных рамок, необходимых для разработки системы.


\begin{image}
    \includegrph[scale=0.23]{Рекомендательная система фильмов.png}
    \caption{Диаграмма Ганта}
    \label{fig:use:case}
\end{image}


\clearpage

\section{Риски}

Таблица рисков помогает заранее определить
и подготовиться к потенциальным проблемам,
которые могут возникнуть в процессе разработки.

\begin{table}[h]
    \centering
    \caption{Перечень основных рисков проекта}
    \label{tab:risk_table}
    \begin{tabular}{|p{3cm}|p{2.5cm}|p{2.5cm}|p{3cm}|p{4cm}|}
        \hline
        \textbf{Название риска} & \textbf{Послед - ствия} & \textbf{Оценка риска} & \textbf{Стратегия реагирования} & \textbf{Мероприятия в рамках стратегии} \\ \hline
        Задержка поставки требований & Срыв сроков проекта & Высокая & Принятие & Создать резерв времени, ускорить сбор требований \\ \hline
        Недостаточная квалификация команды & Низкое качество продукта & Средняя & Снижение & Организовать обучение, привлечь опытных специалистов \\ \hline
        Отказ оборудования & Остановка разработки & Низкая & Принятие & Обеспечить резервное оборудование, регулярное резервирование данных \\ \hline
        Изменение требований заказчиком & Увеличение объёма работ & Высокая & Снижение & Ввести процесс управления изменениями, согласовать ТЗ \\ \hline
        Риск безопасности данных & Утечка информации & Средняя & Принятие & Внедрить меры информационной безопасности, обучить персонал \\ \hline
    \end{tabular}
\end{table}
\clearpage
\section*{\LARGE Вывод}
\addcontentsline{toc}{section}{Вывод}

В результате практической работы
были разработаны диаграмма Ганта и таблица рисков,
что помогает в планировании
и управлении проектом по разработке системы.
Диаграмма Ганта визуализирует задачи, подзадачи и временные рамки,
а таблица рисков помогает заранее определить
и подготовиться к потенциальным проблемам,
которые могут возникнуть в процессе разработки.
Это способствует более эффективному управлению проектом
и повышению шансов на успешное завершение разработки в установленные сроки.


