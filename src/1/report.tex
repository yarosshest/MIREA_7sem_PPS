%\usepackage{hyperref}\section*{\LARGE Цель практической работы}
\addcontentsline{toc}{section}{Цель практической работы}

\textbf{Цель практической работы:}
Сформировать тему работ в семестре.
Сформировать функциональные, пользовательские, нефункциональные и ограничения.
Ответить на вопросы интервью.

\clearpage

\section*{\LARGE Выполнение практической работы}
\addcontentsline{toc}{section}{Выполнение практической работы}

\section{Тема практических работ}\label{sec:theme}
Тема практических работ: Мобильное приложение рекомендации фильмов


\clearpage

\section{Требования}\label{sec:treb}
\subsection{требования к структуре АС в целом}\label{subsec:treb:ob}
\subsubsection{Перечень подсистем}
Сиcтема состоит из следующих подсистем:
\begin{itemize}
    \item База данных;
    \item Модуль рекомендации;
    \item API;
    \item Парсер;
    \item Мобильное приложение.
\end{itemize}

База данных -- xранит информацию о: фильмах, оценках и пользователях.
Модуль рекомендации -- предоставляет личные рекомендации пользователям на основе их оценок.
API -- выполняет функцию для связки всех подсистем.
Мобильное приложение -- предоставляет пользователю интерфейс общения с системой.
Парсер -- собирает новые фильмы и информацию о них.
\subsubsection{Требования к способам и средствам обеспечения информационного взаимодействия компонентов}
Взаимодействие между базой данных и API должно происходить путем прямого подключения через драйвер базы данных.

Взаимодействие между модулем рекомендации может происходить как по следующим средствам: брокер сообщений, RESTapi,
внутри программное взаимодействие.

Взаимодействие между API и мобильным приложением должно происходить через RESTapi и веб-сокеты.

\subsubsection{Требования к характеристикам взаимосвязей создаваемой АС со смежными АС}
Смежными системами к данной являются:
\begin{itemize}
    \item PostgresSQL;
    \item Модели по векторизации текстов;
    \item Решающие деревья рекомендации.
\end{itemize}

Данные системы будут подключаемыми программными модулями для API и модуля рекомендации.
Они должны быть совместимы с целевым языком программирования данных систем: Python 3.9

\subsubsection{Требования к режимам функционирования}

У приложения должно быть 2 режима функционирования:
\begin{itemize}
    \item С рекомендательной системой;
    \item С отключенной рекомендательной системой.
\end{itemize}

В связи со специфичностью работы модуля рекомендации разрабатываемой системы.
Возможны случаи когда система не сможет обрабатывать новые запросы по рекомендации в связи с загруженностью системы.
В данном режиме пользователь должен получить оповещение о проблеме и о том что его рекомендации появиться позже.

\subsubsection{Перспективы развития, модернизации АС}
В будущем возможна создание интерфейса для подключения как к другим систем, так и общения других решений с
разрабатываемой системой.

\subsection{Требования к функциям}\label{subsec:func:treb}

\subsubsection{Время отклика приложения}
Ответ на каждое действие пользователя должно быть не более 30 секунд.
После истечения 30 секунд пользователь должен получить уведомление о проблемах на стороне сервиса.

\subsubsection{Регистрация}
Пользователь должен иметь возможность зарегистрироваться в приложении
Для регистрации используются такие данные:
\begin{itemize}
    \item Уникальный логин;
    \item Пароль длиною более 6 символов.
\end{itemize}
Если данный логин уже используется, то пользователь должен быть уведомлен об этом и мобильное приложение должно
попросить ввести новый уникальный логин.

\subsubsection{Авторизация}
Пользователь должен иметь возможность авторизоваться в приложении.
Для авторизации используются такие данные:
\begin{itemize}
    \item Уникальный логин;
    \item Пароль длиною более 6 символов.
\end{itemize}
Если после ввода данных для аутентификации прошло более 30 секунд и не пришел от сервера, пользователь должен быть
оповещен об ошибке на стороне сервера.

\subsubsection{Поиск фильмов}
Пользователь должен иметь возможность искать фильмы в приложении, по их названию.
Для поиска используется строка длинной не более 512 символов.
Если фильмов не нашлось, пользователь должен получить информацию о том, что фильмов по его запросу нет.

\subsubsection{Просмотр фильмов}
Пользователь должен иметь возможность просматривать информацию о фильме.
А именно:
\begin{itemize}
    \item Фото;
    \item Описание;
    \item Оценка пользователя.
\end{itemize}

\subsubsection{Получение рекомендаций}
Пользователь должен иметь возможность получать рекомендации.
Количество рекомендаций в день может быть ограниченно до 1 в день из-за нагрузки на сервер.
Пользователь должен был оценить как минимум 2 фильма положительно, 2 фильма отрицательно.

\subsubsection{Управление фильмами}
Система должна предоставить возможность для редактирования, добавления, удаления фильма и его параметров.

\subsubsection{Управление доступностью системы рекомендации}
Система должна предоставить возможность для отключения и включения системы рекомендации.

\subsubsection{Обновление библиотеки фильмов}
Система должна предоставить возможность для сбора фильмов из кинопоиска и пред обработки для записи в базу данных.

\subsection{Требования к видам обеспечения}
\subsubsection{Математическое обеспечение}
Система рекомендации должна быть разработанная по данному принципу:
Короткое описание фильма векториризируеться, данный вектор обогащается дополнительными данными о фильме.

Для рекомендации должен использоваться лес решающих деревьев, который будет уникальным для каждого пользователя.
Он будет модифицироваться при каждом запросе рекомендации от пользователя.

\subsubsection{Информационное обеспечение}
В качестве базы данных должна использоваться СУБД PostgresSQL\@.
Формат обмена сообщений между клиентом Android и API должен быть описан в Swagger документе.

\subsubsection{Лингвистическое обеспечение}
Мобильное приложение должно быть написано на языке Kotlin для платформы Android.
api, система рекомендаций и парсинг должен быть написан на языке Python.
Интерфейс должен быть выполнен на русском языке.

\subsubsection{Техническое обеспечение}
Северная часть системы должно разворчится на сервере с 4 ядрами и 16 гигабайтами оперативной памяти на ОС Linux внутри 
Docker.

\subsection{Общие технические требования к АС}
\subsubsection{Требования к численности и квалификации персонала и пользователей АС}
Системой должен управлять 2 системных администратора уровня Middle.
Так же 1 программист уровня Senior.


\subsubsection{Требования к показателям назначения}
Система должна поддерживать 200 одновременно работающих в системе пользователей.
Система должна поддерживать 1000 одновременно выполняемых запросов к серверу.
Система должна поддерживать 1000 одновременно выполняемых запросов к серверу.

\subsubsection{Требования к надежности}
Система должна работать 90\% времени в день.

\subsubsection{Требования к безопасности}
Пароли пользователей должны храниться в зашифрованном виде.

\clearpage

\section{Интервьюирование}
\subsection{Цели и задачи проекта}
\textbf{Какие цели вы хотите достичь с помощью этого проекта?} \par
Я хочу достичь качественной, индивидуальной рекомендации для пользователей. \par
\textbf{Какие конкретные задачи должны быть выполнены?} \par
Точность предсказания для конкретного пользователя должно значительно (от 20\%) превышать существующие решения от
больших кампаний

\subsection{Стейкхолдеры и пользователи}
\textbf{Кто является основными стейкхолдерами проекта?} \par
Cтейкхолдеры это обычные пользователи стримминговых сервисов кино.
Которым не нравиться работа рекомендательной части стримминговых сервисов кино.
\par
\textbf{Какие группы пользователей будут взаимодействовать с системой или продуктом?} \par
Основная группа пользователей это любители кино.
От 12 до 60 лет.

\subsection{Функциональные требования}
\textbf{Какие функции и возможности должны быть реализованы в системе?} \par
Поиск, оценка, рекомендации фильмов.
\par
\textbf{Какие бизнес-процессы должны поддерживаться?} \par
Создание индивидуальной рекомендации для пользователя.
Обновление данных о вышедших фильмах.


\subsection{Нефункциональные требования}
\textbf{Какие требования к производительности, безопасности и доступности системы у вас есть?} \par
Пароли пользователей должны храниться в зашифрованном виде.
Система должна работать 90\% в день.
\par
\textbf{Есть ли особенности в области безопасности данных?} \par
Особенностей нет так как проект позиционируются на отсутствии сбора личной информации.

\subsection{Интеграция и сторонние сервисы}
\textbf{Необходимо ли интегрировать систему с другими приложениями или сторонними сервисами?} \par
Только в будущем если создаться такая потребность у пользователей.
\par
Какие API или протоколы обмена данными следует использовать?
RESTApi и Rabbit.

\subsection{Интерфейс пользователя}
\textbf{Какие требования к пользовательскому интерфейсу (UI) у вас есть?} \par
Время отклика до 30 секунд.
Юзабилити от 70\%.
\par
\textbf{Какие элементы управления и макеты вы предпочли бы использовать?} \par
В проекте необходимо использовать все стандарты Material design.

\subsection{Требования к производительности}
\textbf{Какие ожидания по скорости работы и времени отклика системы?} \par
Время отклика до 30 секунд.
Кроме рекомендательной части она требует не регламентированного времени.
\par
\textbf{Какое количество пользователей системы вы ожидаете?} \par
до 200 одновременных пользователей.


\subsection{Обслуживание и поддержка}
\textbf{Какие требования к технической поддержке и обновлениям после внедрения системы?} \par
Систему должны поддерживать программист и системный администратор.
Обновления должны происходить в рамках обратного отклика от пользователей по увеличению функционала и юзабилити.

\subsubsection{Локализация и интернационализация}
\textbf{Есть ли требования к поддержке разных языков и региональных настроек?} \par
Система может работать только на русском языке, остальные языки требуют координальной переработки системы.

\subsubsection{Уровень безопасности}
\textbf{Какие меры безопасности данных и управления доступом требуются?} \par
Пароли пользователей должны храниться в зашифрованном виде.
База данных использует ролевую модель доступа.

\subsubsection{Требования к документации и обучению}
\textbf{Нужна ли документация для пользователей или администраторов системы?} \par
Документация для пользователй не требуются.
Для администраторов требуются верхне уровневая документация о работе системы и ее возможных ошибках.
\par
\textbf{Требуется ли обучение пользователям?} \par
Обучине не требуется.

