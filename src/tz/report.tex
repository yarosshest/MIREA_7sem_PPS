\section{Общие сведения}

\subsection{Полное наименование автоматизированной системы (АС)
	и её условное обозначение}

\textbf{Полное наименование}:
Мобильное приложение, рекомендательная система фильмов с алгоритмом личной рекомендации фильмов.

\textbf{Условное обозначение}: МП РСФ.

\subsection{Наименование организаций}

В в разработке участвуют следующие организаций:

\begin{itemize}
	\item Шестаков Ярослав Евгеньевич (Исполнитель).
\end{itemize}

\subsection{Перечень документов, на основании которых создается АС}

Для создания АС Исполнителем предъявляются следующие документы:

\begin{itemize}
	%\item Государственный контракт №2124-05-08 от 12.11.2008;
	\item Техническое задание на создание МП РСФ;
	\item Технический проект МП РСФ.
	\item Программа и методика испытаний на МП РСФ;
\end{itemize}

\subsection{Плановые сроки начала и окончания работ по созданию АС}

Дата начала работ: 2 сентября 2024 года

Дата окончания работ: 30 мая 2025 года

\subsection{Общие сведения об источниках и порядке финансирования работ}

Работы по созданию МП РСФ не финансируются, все затраты берет на себя исполнитель.

\section{Цели и назначение создания автоматизированной системы}

\subsection{Цели создания АС}

Цель создания автоматизированной системы рекомендательная система фильмов с алгоритмом личной рекомендации фильмов
заключается в создании экспериментального алгоритма, и реализации взаимодействия пользователя с ним.
В результате создания системы должны быть достигнуты следующие показатели:

\subsubsection{Технические показатели}

Время выдачи рекомендации.
Не более 5 минут для 1 пользователя.

Качество рекомендации.
Точность рекомендаций должна быть больше 80\%.


\subsubsection{Технологические показатели}

Масштабируемость.
Сервис рекомендации должен адаптироваться к увеличению числа пользователей и их меняющимся потребностям.
Архитектура должна позволять внедрять новые функции и увеличивать пропускную способность.

Архитектура данных.
Обеспечивает сбор, хранение и предоставление информации пользователей приложения.


\subsubsection{Критерии оценки достижения целей}

Качество рекомендации должно соответствовать указанным значениям.

Сервис рекомендации должен обеспечивать до 100 одновременных запросов.

Удобство интерфейса проверяется на стадии тестирования
и доработки с привлечением инженеров верификации
в качестве тестовой группы пользователей.

\subsection{Назначение АС}

АС предназначена для автоматизации процесса персонального подбора фильмов,
используя алгоритмы рекомендации, основанные на предпочтениях пользователя.
Основной вид автоматизируемой деятельности --- упрощение процесса
выбора контента и повышение удобства при просмотре фильмов на мобильных устройствах.

АС применяется в следующих объектах автоматизации:

\begin{itemize}
	\item Мобильное приложение для просмотра фильмов --- система предназначена
	для использования конечными пользователями, где формируется персонализированная
	лента рекомендаций фильмов с учётом вкусов и истории просмотра;
	\item Рекомендательная система фильмов --- будет использоваться для анализа
	поведения пользователя, рейтингов и отзывов с целью формирования релевантных
	рекомендаций, что обеспечивает повышение качества пользовательского опыта.
\end{itemize}


\section{Характеристика объекта автоматизации}

\subsection{Основные сведения об объекте автоматизации}

Объектом автоматизации --- является процесс формирования персональных рекомендаций фильмов
для мобильного приложения, направленный на создание и внедрение собственного
экспериментального алгоритма рекомендаций, не зависящего от сторонних платформ.
Данный процесс включает анализ предпочтений и поведения пользователей, а также
разработку и оптимизацию оригинального алгоритма личной рекомендации.

В отличие от подходов, основанных на использовании готовых инструментов и платформ
(таких как Google Recommender API или Amazon Personalize), в данном случае упор делается
на полный цикл контроля над алгоритмом: от сбора и анализа данных до формирования
персонализированных рекомендаций без необходимости обеспечения совместимости
с внешними системами.

Программный компонент (ПК) рекомендательной системы предназначен для предоставления
единых инструментов разработки, тестирования и исполнения экспериментального алгоритма,
минимизируя зависимость от сторонних решений. Это упрощает процесс совершенствования модели
и обеспечивает удобное взаимодействие пользователя с рекомендационной системой.

\subsection{Сведения об условиях эксплуатации
	и характеристиках окружающей среды}

МП РСФ предназначена для использования в офисных условиях
на рабочих станциях инженеров по верификации и проектированию.

Операционные системы:

\begin{itemize}
	\item Linux;
	\item Android.
\end{itemize}

\textbf{Минимальные системные требования:}

\begin{itemize}
	\item Процессор: 2 ГГц, 2 ядра;
	\item ОЗУ: 4 ГБ;
	\item GPU: 3060 GTX и мощёнее;
	\item Дисковое пространство: 5 МБ для установки, 20 ГБ для данных.
\end{itemize}

\textbf{Рекомендуемые системные требования:}

\begin{itemize}
	\item Процессор: 3 ГГц, 16 ядер;
	\item ОЗУ: 32 ГБ;
	\item Дисковое пространство: 50-100 ГБ.
\end{itemize}


\section{Требования к видам обеспечения АС}


\subsection{Требования к математическому обеспечению системы}

Для реализации математического обеспечения рекомендационной системы фильмов
требуются следующие алгоритмы и методы:

\begin{itemize}
	\item Алгоритмы машинного обучения для построения моделей рекомендаций
	(например, коллаборативная фильтрация, факторизация матриц,
	нейронные сети или методы обучения с подсказками);
	\item Методы обработки и анализа пользовательских данных, включая нормализацию,
	фильтрацию шумов, выделение признаков и агрегацию поведенческих метрик;
	\item Алгоритмы обработки естественного языка (NLP) для анализа текстовой информации,
	такой как описания фильмов или отзывы пользователей
	(токенизация, лемматизация, определение ключевых слов и фраз);
	\item Методы оценки качества рекомендаций, включая расчёт метрик точности (Precision, Recall, F1),
	а также использование A/B-тестирования и других экспериментов
	для валидирования эффективности рекомендационного алгоритма.
\end{itemize}

\subsection{Требования к информационному обеспечению системы}

\subsubsection{Состав и структура данных}

Рекомендательная система фильмов оперирует следующими типами данных:
\begin{itemize}
	\item Оценки и отзывы, отражающие мнение пользователей о просмотренных фильмах;
	\item Метаинформация о фильмах (жанр, год выпуска, актёрский состав, страна производства);
	\item Конфигурационные файлы, содержащие параметры рекомендационного алгоритма (например, весовые коэффициенты, гиперпараметры моделей).
\end{itemize}

Данные структурированы по логическим модулям, чтобы обеспечить удобный доступ,
систематизацию и обработку необходимых для формирования персональных рекомендаций сведений.

\subsubsection{Организация данных}

Информация хранится в текстовых форматах (например, JSON или CSV), доступных для чтения и редактирования.
Файлы могут быть размещены на локальном устройстве или в облачном хранилище.
Конфигурационные файлы содержат настройки и параметры алгоритма,
обеспечивая корректную работу всей системы.

\subsubsection{Информационный обмен}

Взаимодействие между модулями системы (сбор данных, предобработка, обучение модели, формирование рекомендаций)
должно быть организовано так, чтобы обеспечить непрерывный поток информации и правильную
последовательность выполнения операций. Это необходимо для корректной интеграции данных о пользователях,
фильмах и результатах работы рекомендационного алгоритма.

\subsubsection{Информационная совместимость}

Система должна обеспечивать совместимость со стандартными форматами данных,
используемыми для описания фильмов (например, данные из внешних API-источников),
а также с форматами пользовательских данных, экспортируемыми из различных
сервисов или приложений. Это необходимо для расширения функциональности,
масштабируемости и интеграции со сторонними источниками данных.

\subsubsection{Классификаторы и справочники}

Для систематизации информации о фильмах и аудитории необходимо использование
справочников и классификаторов (например, справочник жанров, классификатор возрастных групп,
справочник стран производства). Это обеспечивает более точную фильтрацию и персонализацию рекомендаций.

\subsubsection{Системы управления базами данных}

Использование реляционных или нереляционных СУБД может потребоваться
при усложнении системы или увеличении объёмов данных.


\subsubsection{Контроль и восстановление данных}

Система должна вести логи работы, фиксируя каждую операцию, а также возникающие ошибки.

\subsection{Требования к лингвистическому обеспечению АС}

Поддерживаемые языки.
Мобильное приложение должно быть написано на языке Kotlin для платформы Android.
api, система рекомендаций и парсинг должен быть написан на языке Python.

Интерфейс.
Интерфейс должен быть выполнен на русском языке.

\subsection{Требования к программному обеспечению АС}

\subsubsection{Состав ПО}

Сиcтема состоит из следующих подсистем:
\begin{itemize}
	\item База данных;
	\item Модуль рекомендации;
	\item API;
	\item Парсер;
	\item Мобильное приложение.
\end{itemize}

База данных -- xранит информацию о: фильмах, оценках и пользователях.
Модуль рекомендации -- предоставляет личные рекомендации пользователям на основе их оценок.
API -- выполняет функцию для связки всех подсистем.
Мобильное приложение -- предоставляет пользователю интерфейс общения с системой.
Парсер -- собирает новые фильмы и информацию о них.

\subsubsection{Выбор ПО}

Язык программирования: Python (версии 3.10 и выше).

Библиотеки Python:

\begin{itemize}
	\item scikit-learn;
	\item fastapi;
	\item catboost;
	\item nltk;
	\item pymystem3;
	\item navec.
\end{itemize}

\subsubsection{Разрабатываемое ПО}

Реализация алгоритма личной рекомендации фильмов, основанный на текстовом описании фильма и оценках пользователя.
С мобильным приложением в виде интерфейса пользователя.

\subsubsection{Покупные средства}

Использование библиотек с открытым исходным кодом.

\subsubsection{Архитектура ПО}

Модель C4 описывает архитектуру системы на четырёх уровнях:
контекста, контейнеров, компонентов и кода.
На основе этой модели можно создать следующие диаграммы:

\paragraph{Диаграмма контекста системы}

В модели C4 диаграмма контекста системы (System Context Diagram)
представляет собой самый высокий уровень абстракции системы,
показывая взаимодействие системы с внешними пользователями (акторами)
и системами. Она дает общее представление о том,
кто и как взаимодействует с системой,
не углубляясь в детали внутренней реализации.
   
\begin{image}
	\includegrph[scale=0.15]{C41.png}
	\caption{Диаграмма контекста системы}
	\label{fig:c4:system:context}
\end{image}

\clearpage

Пользователь взаимодействует с системой через интерфейс мобильного приложения для: поиска, оценки фильмов, рекомендаций.\par


\paragraph{Диаграмма контейнеров}

Диаграмма контейнеров представляет следующий уровень детализации модели C4.
Она показывает внутреннюю структуру системы,
отображая основные контейнеры (программные компоненты) системы,
такие как приложения, базы данных, внешние системы, и их взаимодействия.
   
\begin{image}
	\includegrph[scale=0.1]{C42.png}
	\caption{Диаграмма компонентов}
	\label{fig:c4:container}
\end{image}

\clearpage

\paragraph{Диаграмма компонентов}

Диаграмма компонентов модели C4 детализирует
каждый контейнер на более глубоком уровне,
отображая его внутренние программные компоненты и взаимодействия между ними.
Это позволяет увидеть, как функционирует каждая часть контейнера
и какие компоненты обеспечивают выполнение основных функций.

\begin{image}
	\includegrph[scale=0.07]{C43.png}
	\caption{Диаграмма компонентов}
	\label{fig:c4:components}
\end{image}

\clearpage

\subsection{Требования к техническому обеспечению АС}

\subsection*{Технические средства.}
Персональные сервера на ОС Linux.

\subsection*{Функциональные характеристики}

\subsubsection*{Общие возможности системы}
\begin{itemize}
	\item Обеспечение стабильной работы при нагрузке до 1\,000 одновременно активных пользователей.
	\item Обработка до 10 запросов в секунду при средней задержке не более 200 миллисекунд.
	\item Поддержка масштабирования до 100\,000 зарегистрированных пользователей.
	\item Интеграция с внешними API для получения данных с частотой обновления до 1 запроса в 10 секунд.
\end{itemize}

\subsubsection*{Требования к серверной части}
\begin{itemize}
	\item Поддержка REST API с пропускной способностью не менее 500 Мбит/с.
	\item Обеспечение устойчивости к сбоям с доступностью 99.9\% (время простоя не более 8 часов в год).
	\item Шифрование данных с использованием алгоритма AES-256.
	\item Возможность обработки до 1\,000 транзакций в минуту.
\end{itemize}

\subsubsection*{Требования к базе данных}
\begin{itemize}
	\item Реляционная база данных (например, PostgreSQL) с максимальным размером таблиц до 10 ТБ.
	\item Среднее время выполнения сложных запросов: не более 500 миллисекунд.
	\item Хранение информации о 100\,000 фильмов, включая:
	\begin{itemize}
		\item Название, описание, жанры, рейтинг, постеры.
		\item Данные о пользователях: профили, история просмотров, предпочтения.
		\item Логи взаимодействий объёмом до 100 ГБ в месяц.
	\end{itemize}
	\item Репликация данных в режиме реального времени с задержкой не более 1 секунды.
\end{itemize}

\subsubsection*{Требования к клиентской части}
\begin{itemize}
	\item Совместимость с устройствами под управлением Android версии 8.0+ .
	\item Максимальный размер установочного файла: 50 МБ.
	\item Потребление оперативной памяти не более 300 МБ при активной работе.
	\item Уведомления с задержкой доставки не более 5 секунд.
	\item Возможность работы в режиме офлайн с хранением до 500 записей в локальной базе данных.
\end{itemize}

\subsubsection*{Требования к оборудованию}
\begin{itemize}
	\item \textbf{Серверная часть:}
	\begin{itemize}
		\item Процессор: минимум 8 ядер с частотой 2.5 ГГц.
		\item Оперативная память: минимум 32 ГБ.
		\item Хранилище: SSD от 1 ТБ с возможностью масштабирования до 10 ТБ.
		\item Сетевое соединение: симметричный канал от 1 Гбит/с.
	\end{itemize}
	\item \textbf{Клиентская часть:}
	\begin{itemize}
		\item Минимальные устройства: смартфоны с 2 ГБ оперативной памяти и 16 ГБ свободного места.
		\item Экран: поддержка разрешения 720p и выше.
	\end{itemize}
\end{itemize}

\subsubsection*{Требования к производительности системы}
\begin{itemize}
	\item Время отклика пользовательского интерфейса: не более 100 миллисекунд.
	\item Средняя загрузка процессора сервера: не более 70\% при пиковой нагрузке.
	\item Возможность восстановления системы после сбоя в течение 10 минут.
	\item Среднее время обновления данных в базе: не более 1 секунды.
\end{itemize}

\subsection{Требования к метрологическому обеспечению АС}

Показатели метрологического обеспечения.
Точность  должна соответствовать 80\%.

Методы измерений.
Методы оценки качества рекомендаций, включая расчёт метрик точности (Precision, Recall, F1).

Средства измерений и контроля.
Запуск на размеченных данных.

\subsection{Требования к организационному обеспечению АС}

Организация функционирования.
АС запускается и обслуживается системными администраторами
без необходимости постоянного сопровождения.

Организация при сбоях, отказах и авариях.
Автоматическое логирование ошибок.

Нормативные документы.
Обеспечение разработчиков инструкциями по использованию инструментов
для разработки и тестирования.

\subsection{Требования к методическому обеспечению АС}

Для обеспечение документацией разработчики
должны быть обеспечены необходимыми нормативными
и методическими документами для правильного выполнения работ.

\section{Общие технические требования к АС}

\subsection{Требования к численности
	и квалификации персонала и пользователей АС}

Для эксплуатации системы требуется один системный администратор
для запуска сервиса и контроля производительности.

Пользователю требуется минимальное обучение для работы с интерфейсом.

\subsection{Требования к показателям назначения АС}

АС должна обеспечивать:

\begin{itemize}
	\item Точность предсказаний не менее 80\%.
	\item Время выполнения рекомендации не должно превышать 5 минут.
\end{itemize}

\subsection{Требования к надежности}

АС должна обеспечивать корректное выполнение рекомендации
при каждом запуске по запросу пользователя.
Вероятность аварийного завершения работы системы должна быть
не более одного случая на 1000 запусков.

При возникновении ошибок или сбоев система должна отправлять код ошибки
с выдачей соответствующего сообщения об ошибке.
Все ошибки логируются для последующего анализа и устранения.

\subsection{Требования по безопасности}

Данные авторизации пользователей должны быть зашифрованны.

\subsection{Требования к эргономике и технической эстетике}

Интерфейс системы должен быть простым,
обеспечивать доступ ко всем необходимым функциям
с минимальным количеством действий.

\subsection{Требования к транспортабельности для подвижных АС}

Требования к транспортабельности для подвижных АС
не предъявляется.

\subsection{Требования к эксплуатации, техническому обслуживанию,
	ремонту и хранению компонентов АС}

Требования к эксплуатации, техническому обслуживанию,
ремонту и хранению компонентов АС не предъявляется.

\subsection{Требования к защите информации от несанкционированного доступа}

Требования к защите информации от несанкционированного доступа
не предъявляется.

\subsection{Требования по сохранности информации при авариях}

Требования по сохранности информации при авариях не предъявляется.

\subsection{Требования к защите от внешних воздействий}

Требования к защите от внешних воздействий не предъявляется.

\subsection{Требования к патентной чистоте и патентоспособности}

Система должна соответствовать требованиям патентной чистоты.
Перед разработкой необходимо провести патентные исследования
на наличие схожих решений и патентов
для избежания нарушений прав на интеллектуальную собственность.

\subsection{Требования по стандартизации и унификации}

Требования по стандартизации и унификации не предъявляется.

\subsection{Дополнительные требования}

Дополнительные требования не предъявляется.

\section{Состав и содержание работ по созданию автоматизированной системы}

Разработка АС включает следующие этапы:

\begin{longtable}{|p{6cm}|p{5cm}|p{4cm}|}
	\caption{\leftline{Этапы работ}} \label{table:stages} \\
	\hline
	\textbf{Задача}
	& \textbf{Подзадача}
	& \textbf{Время выполнения} \\
	\hline
	\endfirsthead
	\conttable{table:stages} \\
	\hline
	\textbf{Задача}
	& \textbf{Подзадача}
	& \textbf{Время выполнения} \\
	\hline
	\endhead
	1 Проектирование системы
	& Определение требований
	& 5 дней \\ \hline

	2 Разработка базы данных
	& Проектирование архитектуры
	& 5 дней \\ \hline

	3 Разработка API
	& 3.1 Реализация API для просмотра фильмов
	& 3 дней \\ \hline

	& 3.2 Реализация API для поиска фильмов
	& 3 дней \\ \hline

	& 3.3 Реализация API для оценок фильмов
	& 4 дней \\ \hline

	4 Разработка мобильного приложения
	& 4.1 Интерфейс для просмотра фильмов
	& 4 дней \\ \hline

	& 4.2 Интерфейс для поиска фильмов
	& 4 дней \\ \hline

	& 4.3 Интерфейс для оценок фильмов
	& 4 дня \\ \hline

	5 Реализация системы рекомендаций
	&
	& 10 дней \\ \hline

	6 Интеграция системы рекомендаций с API
	&
	& 3 дней \\ \hline

	7 Тестирование системы
	&
	& 7 дня \\ \hline

	8 Развертывание системы
	&
	& 3 дня \\ \hline

	9. Развертывание системы
	&
	& 5 дня \\ \hline
\end{longtable}

\section{Порядок разработки автоматизированной системы}

\subsection{Порядок организации разработки АС}

Разработка AC осуществляется поэтапно,
начиная с подготовки и анализа требований,
проектирования архитектуры и заканчивая внедрением
и сдачей готовой системы заказчику.
Каждый этап включает плановые проверки и согласования с заказчиком.
Проектная команда включает аналитиков, разработчиков,
тестировщиков и специалистов по документации.
Заказчик участвует в согласовании технического задания,
промежуточных отчетов, демонстраций функционала и финальной приемке системы.

\subsection{Перечень документов и исходных данных для разработки АС}

\begin{itemize}
	\item Техническое задание на разработку АС;
	\item Cатьи по рекомендательным системам основанным на тексте;
	\item Документация catboost;
	\item Нормативные документы,
		регламентирующие стандарты разработки ПО и интерфейсов.
\end{itemize}

\subsection{Перечень документов, предъявляемых по окончании этапов работ}

На каждом этапе разработки будут предоставлены следующие документы:

\begin{enumerate}
	\item Подготовительный этап:
		Аналитический отчет по требованиям, утвержденное техническое задание.
	\item Проектирование системы:
		Документация по архитектуре системы,
		прототип интерфейса, документ проектирования модулей.
	\item Разработка модулей:
		Отчет о готовности модулей, описание их функционала,
		результаты промежуточных тестов.
	\item Тестирование системы:
		Отчет о тестировании, включая результаты модульного,
		интеграционного и системного тестирования.
	\item Документация:
		Руководство пользователя, техническая документация,
		отчет о проведенных испытаниях.
	\item Внедрение и обучение пользователей:
		Отчет о внедрении, программа обучения пользователей.
	\item Сдача системы заказчику:
		Итоговый отчет о готовности системы, акт приема-передачи.
\end{enumerate}

\subsection{Порядок проведения экспертизы технической документации}

Техническая документация, включая отчет о тестировании
и руководство пользователя, проходит экспертизу у специалистов заказчика
и специалистов компании-разработчика.
Экспертиза проводится после завершения этапа разработки документации,
результаты утверждаются обеими сторонами. 

\subsection{Перечень макетов, порядок их разработки, испытаний и документации}

Перечень макетов, порядок их разработки, испытаний
и документации не предъявляется.

\subsection{Порядок разработки, согласования
	и утверждения плана совместных работ по разработке АС}

План совместных работ согласуется на этапе подготовки проекта
и утверждается обеими сторонами.
Включает сроки и ответственность за выполнение каждого этапа разработки,
ключевые контрольные точки и порядок взаимодействия разработчиков и заказчика. 

\subsection{Порядок разработки, согласования
	и утверждения программы работ по стандартизации}

Программа работ по стандартизации разрабатывается на основе требований ГОСТ
и внутренних стандартов заказчика.
Программа включает меры по обеспечению совместимости форматов,
унификации интерфейсов и стандартов качества ПО.
Программа согласуется с заказчиком
и утверждается на начальном этапе разработки.

\subsection{Требования к гарантийным обязательствам разработчика}

Разработчик предоставляет гарантию на исправление ошибок
и поддержку системы в течение 12 месяцев после сдачи и приемки МП РСФ.
В гарантийные обязательства входит исправление критических ошибок
в обновление документации и консультирование заказчика.

\subsection{Порядок проведения технико-экономической оценки разработки АС}

Технико-экономическая оценка включает анализ затрат на разработку,
внедрение и поддержку МП РСФ.
Оценка проводится по завершении проектирования и тестирования системы,
чтобы определить соотношение затрат и пользы,
получаемой заказчиком от внедрения системы.

\subsection{Порядок разработки, согласования
	и утверждения программы метрологического обеспечения,
	программы обеспечения надежности, программы эргономического обеспечения}

Программа метрологического обеспечения
разрабатывается с целью контроля точности предсказаний
и корректности модели рекомендации.
Включает метрики предсказаний моделей.
  
Программа обеспечения надежности содержит мероприятия
по тестированию на устойчивость к отказам,
механизмы восстановления данных и обеспечение сохранности данных при сбоях.

Программа эргономического обеспечения разрабатывается
для проверки удобства и интуитивности интерфейса МП РСФ.
Включает тестирование интерфейса на восприятие
и анализ действий пользователя для оптимизации взаимодействия.

Каждая из программ согласуется с заказчиком
и утверждается на этапе проектирования системы.

\section{Порядок контроля и приемки автоматизированной системы}

\subsection{Виды, состав и методы испытаний АС и её составных частей}

Испытания автоматизированной системы МП РСФ проводятся в несколько этапов,
чтобы подтвердить работоспособность
и соответствие системы заявленным требованиям:

\subsubsection{Модульные испытания}

Испытания происходит над отдельными модулями.

Тестирование включает проверку функциональности
и корректности каждого модуля по заранее подготовленным тест-кейсам.
Выполняется проверка корректности работы на отдельных наборах тестовых пользователей.
  
\subsubsection{Интеграционные испытания}

Проверка взаимодействия всех модулей системы.

Проводится последовательное тестирование связи между модулями.
Проверяется корректность передачи данных, синхронизация
и совместимость модулей.

\subsubsection{Системные испытания}

Проверка всей АС как единого целого.

Системное тестирование проводится на реальных данных,
предоставленных заказчиком.
Проверяется полная функциональность системы,,
скорость выполнения, качество предсказаний, надежность системы.

\subsubsection{Приемочные испытания}

Комплексное тестирование системы заказчиком.

Испытания проводятся на реальных пользователях для оценки корректности
и производительности системы в рабочих условиях.
Проверяется соответствие системы всем заявленным в ТЗ характеристикам.

\subsection{Общие требования к приемке работ,
	порядок согласования и утверждения приемочной документации}

Для приемки работ проводятся следующие мероприятия:

\begin{itemize}
	\item После успешного завершения каждого этапа испытаний создается отчет,
		содержащий результаты тестов,
		выявленные ошибки и выполненные корректировки;
	\item По завершении всех этапов испытаний и доработок,
		проведенных на основе выявленных замечаний,
		составляется итоговый акт о готовности системы к приемке;
	\item Итоговая приемка включает подтверждение соответствия системы
		всем функциональным, производственным и техническим требованиям,
		указанным в техническом задании;
	\item Документация по итогам приемки, включая акты и протоколы испытаний,
		согласуется с заказчиком и утверждается обеими сторонами.
\end{itemize}

\subsection{Статус приемочной комиссии}

Приемочная комиссия имеет \textbf{ведомственный статус}
и включает представителей заказчика,
представителей организации-разработчика, а также, при необходимости,
независимых экспертов в области проектирования и верификации ИС.

\section{Требования к составу и содержанию работ по подготовке
	объекта автоматизации к вводу автоматизированной системы в действие}

\subsection{Создание условий функционирования объекта автоматизации}

Осуществить проверку системного окружения
и установку требуемого ПО для корректной работы АС,
включая библиотеки Python и необходимые пакеты.

Предоставить тестовый набор данных,
а также создать среду для тестирования системы в условиях,
приближенных к рабочим, для дополнительной отладки и проверки.

\subsection{Проведение необходимых организационно-штатных мероприятий}

Выделить ответственного за ввод в эксплуатацию
и обеспечение непрерывного функционирования системы.

Ответственные лица от организации-заказчика
и организации-разработчика координируют ввод системы
и решение возможных проблем.

Организовать базовые службы технической поддержки на период эксплуатации,
включая помощь с обновлением ПО и устранением ошибок.

\subsection{Порядок обучения персонала и пользователей АС}

Провести обучение для системных администраторов,
которые будут использовать систему.
Обучение должно включать базовые и расширенные функции системы,
настройку конфигураций, обработку исходных данных и анализ отчетов.

Подготовить учебные пособия, включающие руководство пользователя,
пошаговые инструкции по запуску и настройке системы.

\section{Требования к документированию}

\subsection{Перечень подлежащих разработке документов}

\begin{itemize}
	\item Техническое задание (ТЗ);
	\item Проектная документация (проект, спецификации);
	\item Программа и методика испытаний;
	\item Руководства пользователя и администраторов;
	\item Отчеты о внедрении и эксплуатации системы.
\end{itemize}

\subsection{Вид представления и количество документов}

\subsubsection{Формат}

Все документы предоставляются в электронном формате (PDF),
удобном для хранения и печати,
а также в текстовом формате (Word или аналог) для возможных правок.

\subsubsection{Количество экземпляров}

По одному электронному экземпляру каждой документации
для заказчика и для внутреннего хранения у разработчика.

При необходимости печатные версии, по одному экземпляру Технического задания,
Руководства пользователя и Акта приемки для заказчика.

\subsection{Требования по использованию ЕСКД и ЕСПД}

Документация должна соответствовать требованиям
Единой системы конструкторской документации (ЕСКД)
и Единой системы программной документации (ЕСПД):

\begin{itemize}
	\item ЕСКД: Применяется для проектной документации,
		где должны быть соблюдены стандарты по оформлению, обозначениям
		и структуре документа.
		Это включает требования к оформлению чертежей,
		блок-схем и структурных схем модулей системы.
	\item ЕСПД: Используется для программной и эксплуатационной документации.
		Все описания, инструкции и руководства должны быть оформлены
		в соответствии с требованиями ЕСПД, включая единообразие терминов,
		нумерацию разделов и описание функциональных возможностей.
\end{itemize}

Соблюдение стандартов ЕСКД и ЕСПД необходимо
для обеспечения удобства использования и единообразия документации,
что облегчает эксплуатацию, обслуживание и сопровождение АС.

\section{Источники разработки}

\subsection{Перечень документов и информационных материалов}

ГОСТ 34.601-90 (Автоматизированные системы. Стадии создания)
--- стандарт регламентирует этапы создания автоматизированных систем
и общий порядок их разработки.

ГОСТ Р ИСО/МЭК 25010-2015 (Системная и программная инженерия.
Модели качества систем и программных продуктов) --- стандарт содержит модели
и критерии оценки качества программного обеспечения,
используемые для разработки и тестирования АС.


Внутренняя документация заказчика --- содержит требования к формату отчетов
и стандартам оформления документации,
используемые при подготовке выходных данных и отчетов АС.

\subsection{Использование источников}

ГОСТ 34.601-90 обеспечивает основу для организации этапов разработки
и последовательности работ.

ГОСТ Р ИСО/МЭК 25010-2015 применяется
для определения показателей качества АС, включая надежность,
производительность, совместимость и эргономику.


Внутренняя документация заказчика направлена
на стандартизацию выходной документации,
формирование отчетов и описание форматов файлов,
что позволяет удовлетворить требования пользователя
и поддерживать единообразие данных.

