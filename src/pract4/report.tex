\section{Средства разработки для программного средства}

\subsection{Язык программирования}
\textbf{Python} — используется для серверной части системы, которая отвечает за обработку данных и реализацию алгоритмов рекомендации. Python поддерживает множество библиотек для машинного обучения и работы с базами данных, таких как CatBoost и SqlAlchemy.

\textbf{Kotlin} — основной язык программирования для Android-приложений, обеспечивающий современный и эффективный подход к разработке мобильных интерфейсов. Kotlin, благодаря своей лаконичности и полной совместимости с Java, позволяет создавать стабильные и поддерживаемые приложения.

\subsection{Алгоритм машинного обучения}
\textbf{CatBoost} — это библиотека градиентного бустинга, используемая для построения дерева решений в системе рекомендаций фильмов. CatBoost эффективно работает с категориальными данными и предлагает высокую точность предсказаний при небольшом времени обучения, что делает его идеальным выбором для рекомендательной системы.

\subsection{Интеграционная среда разработки}
\textbf{Pycharm} — это мощная IDE, используемая для разработки серверной части приложения на Python. Она поддерживает отладку кода, тестирование и управление зависимостями, что помогает оптимизировать процесс разработки рекомендательной системы.

\textbf{Android Studio} — основная интегрированная среда разработки для Android-приложений, которая поддерживает Kotlin и Java. С её помощью создается пользовательский интерфейс мобильного приложения, которое взаимодействует с сервером через Rest API.

\subsection{Фреймворк для серверной части}
\textbf{FastAPI} — это современный и высокопроизводительный веб-фреймворк для Python, используемый для создания Rest API серверной части системы. FastAPI обеспечивает быструю разработку и лёгкую масштабируемость, что делает его идеальным выбором для высоконагруженных приложений.

\subsection{База данных}
\textbf{PostgreSQL} — это реляционная база данных, используемая для хранения данных о фильмах и предпочтениях пользователей. PostgreSQL поддерживает работу с большими объемами данных и обладает высоким уровнем производительности, что делает её подходящей для рекомендательных систем.

\subsection{ORM для работы с базой данных}
\textbf{SQLAlchemy} — это мощная ORM-библиотека для Python, которая используется для взаимодействия с базой данных Postgres. Она упрощает работу с базой данных, обеспечивая легкую интеграцию с Python-кодом, и предоставляет удобные инструменты для работы с запросами.

\subsection{Фреймворк для тестирования}
\textbf{Pytest} предоставляет удобные инструменты для написания и выполнения модульных и интеграционных тестов в Python. Он поддерживает автоматизацию тестирования и является важным компонентом для обеспечения стабильности серверной части системы.

\subsection{Система управления версиями}
\textbf{Git} — используется для контроля версий как серверной части, так и мобильного приложения. Интеграция с GitHub или GitLab позволяет эффективно отслеживать изменения кода, организовать работу команды и автоматизировать процессы CI/CD.


\section{Предлагаемая архитектура системы}

Основной принцип архитектуры системы рекомендации
--- \textbf{модульность}.
Программа состоит из отдельных компонентов (модулей),
которые отвечают за конкретные функции:
парсинг, рекомендация, основная логика и мобильное приложение.
Это позволяет легко изменять или добавлять новые функции
без необходимости переписывать всю систему.\par
Также в программном продукте используется принцип
\textbf{разделения ответственности}
(Separation of Concerns или SoC), где каждому модулю назначена
своя ответственность, что снижает взаимозависимость между компонентами.
Это делает систему более поддерживаемой и упрощает тестирование и отладку.

\subsection{Модуль мобильное приложение}
Пердоставляет пользователю возможность общения, регистрации, просмотра, оценки и получения рекомендаций.

\subsection{Модуль парсинга Кинопоиска}
Собирает информацию о фильмах с Кинопоиска, записывает начальные данные в базу данных.
После обрабатывает описания фильмов и векторизует, записывая результаты в базу данных.

\subsection{Модуль основной логики (API)}
Реализует взаимодействие пользователя с базой данных и системой рекомендации.

\subsection{Модуль рекосиндации}
Модуль предоставляет индивидуальные рекомендации для конкретного пользователя.

\section{Архитектурную диаграмму в модели C4}
Модель C4 описывает архитектуру системы на четырёх уровнях:
контекста, контейнеров, компонентов и кода.
На основе этой модели можно создать следующие диаграммы
для программы-конвертера:

\subsection{Диаграмма контекста системы}
В модели C4 диаграмма контекста системы (System Context Diagram)
представляет собой самый высокий уровень абстракции системы,
показывая взаимодействие системы с внешними пользователями (акторами)
и системами.
Она дает общее представление о том,
кто и как взаимодействует с системой,
не углубляясь в детали внутренней реализации.

\begin{image}
    \includegrph[scale=0.1]{С41}
    \caption{Диаграмма контекста системы}
    \label{fig:c4:system:context}
\end{image}

Пользователь взаимодействует с системой через интерфейс мобильного приложения для авторизации, регистрации, просмотра и
оценке фильмов.\par
Кинопоиск является источником данных о фильмах для системы.
\clearpage

\subsection{Диаграмма контейнеров}

Диаграмма контейнеров представляет следующий уровень детализации модели C4.
Она показывает внутреннюю структуру системы,
отображая основные контейнеры (программные компоненты) системы,
такие как приложения, базы данных, внешние системы, и их взаимодействия.

\begin{image}
    \includegrph[scale=0.1]{С42}
    \caption{Диаграмма контейнеров}
    \label{fig:c4:container}
\end{image}


\subsection{Диаграмма компонентов}

Диаграмма компонентов модели C4 детализирует
каждый контейнер на более глубоком уровне,
отображая его внутренние программные компоненты и взаимодействия между ними.
Это позволяет увидеть, как функционирует каждая часть контейнера
и какие компоненты обеспечивают выполнение основных функций.

\begin{image}
    \includegrph[scale=0.1]{С43}
    \caption{Диаграмма компонентов}
    \label{fig:c4:components}
\end{image}


\section{Масштабируемость системы}

Масштабируемость системы заключается в её способности обрабатывать все большее кол-во фильмов и пользователей.

\subsection{Вертикальная масштабируемость}

При увеличении мощности оборудования
(например, процессора и оперативной памяти)
программа сможет обрабатывать большее кол-во запросов о рекоминдации.

\subsection{Масштабируемость за счёт модульности}

Каждый компонент программы
может быть расширен или заменён,
что позволит легко добавлять новые возможности
без изменения существующего кода.

\section{Инструменты, используемые для реализации компонентов архитектуры}

\subsection{JWT}
Формат хранения данных авторизации пользователей , в системе компонент JWT производит авторизацию пользователей и
передает информацию о них дальше.

\subsection{БД}
Компонент взаимодействия с БД через ORM SQLAlhemy, через асинзронные вызовы.

\subsection{API}
Компонент с основной логикой системы , обеспечивает взаимодействие основных компонентов.

\subsection{Система рекоминдации}
Создает рекомендации на основе оценок пользователя.

\subsection{Обработчик}
Обработчик векторизует данны о фильмах.


\subsection{Сборщик}
Собирает информацию о фильмах с сайтов.

\clearpage

