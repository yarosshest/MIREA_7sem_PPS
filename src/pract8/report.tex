\section{Описание используемых технологий и их обоснование}

\subsection{Язык программирования}

Kotlin c html являются базовыми языками для разработки мобильных приложений на android.
Разметка может быть выполнена на любом из данных языков.
\subsection{Библиотека Compose}

Jetpack Compose — это современная декларативная библиотека для создания пользовательских интерфейсов на языке
программирования Kotlin в экосистеме Android.
Она разрабатывается Google как часть набора инструментов Jetpack и представляет собой более упрощенный и
интуитивно понятный подход к разработке UI по сравнению с классическими XML-макетами.
Compose позволяет описывать интерфейсы декларативно, что облегчает их создание, тестирование и модификацию.

\clearpage
\subsection{Меж-экранное взаимодействие}
Давайте рассмотрим меж-экранные взаимодействия в мобильном клиент-серверном приложении.
\begin{itemize}
	\item Экран авторизации:
	\begin{itemize}
		\item Пользователь вводит свои учетные данные (логин и пароль).
		\item Приложение отправляет запрос на сервер
		для проверки подлинности.
		\item В случае успешной аутентификации пользователь
		перенаправляется на главный экран приложения.
		\item Пользователь может перейти на экран регистрации.
		\item Учетные данные могут быть сохранены в приложении.
	\end{itemize}
	\item Экран регистрации:
	\begin{itemize}
		\item Пользователь вводит свои учетные данные (логин и пароль).
		\item Приложение отправляет запрос на сервер
		для проверки.
		\item В случае успешной регистрации пользователь
		перенаправляется на экран авторизации.
	\end{itemize}
	\item Главный экран:
	\begin{itemize}
		\item Пользователь может выбрать фрагмент поиска фильма.
		\item Пользователь может выбрать фрагмент личных рекомендаций.
	\end{itemize}
	\item Фрагмент поиска фильма:
	\begin{itemize}
		\item Пользователь может найти фильм по его названию.
		\item Пользователь может перейти в экран фильма.
	\end{itemize}
	\item Фрагмент личных рекомендаций:
	\begin{itemize}
		\item Пользователь получает личные рекомендации на основе выставленных лайков.
		\item Пользователь может перейти в экран фильма.
	\end{itemize}
	\item Экран фильма:
	\begin{itemize}
		\item Пользователь может прочитать данные о фильме.
		\item Пользователь может оценить фильм как понравившийся и как не понравившийся.
	\end{itemize}
\end{itemize}

В каждом из этих сценариев меж-экранные взаимодействия включают отправку запросов на сервер для получения или
обновления данных, а также отображение результата операции на мобильном устройстве пользователя.
Это обеспечивает плавное и эффективное взаимодействие между клиентским и серверным компонентами приложения.

\subsection{Создание экранов приложения}

В современном мире мобильных и веб-приложений, пользовательский интерфейс (UI) играет ключевую роль в восприятии и
успешности продукта.
Создание экранов приложения — это не просто процесс проектирования и программирования отдельных элементов интерфейса.
Это искусство, требующее глубокого понимания потребностей пользователя, принципов дизайна и технических возможностей
платформы.

Экран авторизации: важный компонент для обеспечения безопасности и персонализированного доступа к системе,
разработанный в рамках данной курсовой работы.

\begin{image}
	\includegrph[scale=0.28]{login}
	\caption{Экран авторизации}
	\label{engineering:login}
\end{image}

\clearpage

Экран регистрации: интерфейс для создания новых учетных записей, обеспечивающий удобный и безопасный процесс
регистрации пользователей.

\begin{image}
	\includegrph[scale=0.28]{signup}
	\caption{Экран регистрации}
	\label{engineering:signup}
\end{image}
Экран поиска: удобный инструмент для быстрого и эффективного нахождения нужной информации в системе.
\begin{image}
	\includegrph[scale=0.28]{find}
	\caption{Экран поска}
	\label{engineering:find}
\end{image}
\clearpage
Экран оценки: простой и наглядный способ выразить свое мнение о фильме, выбрав либо лайк, либо дизлайк.

\begin{image}
	\includegrph[scale=0.28]{film}
	\caption{Экран фильма}
	\label{engineering:film}
\end{image}

Экран рекомендаций: персонализированные рекомендации фильмов, основанные на оценках и предпочтениях пользователей,
чтобы помочь вам открыть новые кинематографические шедевры.
\begin{image}
	\includegrph[scale=0.40]{recom}
	\caption{Экран рекоминдаций}
	\label{engineering:recom}
\end{image}


\section*{\LARGE Вывод}
\addcontentsline{toc}{section}{Вывод}

Разработанный интерфейс приложения предоставляет пользователю способ работы с программой.



