%\usepackage{hyperref}\section*{\LARGE Цель практической работы}
\addcontentsline{toc}{section}{Цель практической работы}

\textbf{Цель практической работы:}
Сформировать тему работ в семестре.
Сформировать функциональные, пользовательские, нефункциональные и ограничения.
Ответить на вопросы интервью.

\clearpage

\section*{\LARGE Выполнение практической работы}
\addcontentsline{toc}{section}{Выполнение практической работы}

\section{Тема практических работ}\label{sec:theme}
Тема практических работ: Мобильное приложение рекомендации фильмов

\section{Требования}\label{sec:treb}
\subsection{требования к структуре АС в целом}\label{subsec:treb:ob}
\subsubsection{Перечень подсистем}
Сиcтема состоит из следующих подсистем:
\begin{itemize}
    \item База данных;
    \item Модуль рекомендации;
    \item API;
    \item Парсер;
    \item Мобильное приложение.
\end{itemize}

База данных -- xранит информацию о: фильмах, оценках и пользователях.
Модуль рекомендации -- предоставляет личные рекомендации пользователям на основе их оценок.
API -- выполняет функцию для связки всех подсистем.
Мобильное приложение -- предоставляет пользователю интерфейс общения с системой.
Парсер -- собирает новые фильмы и информацию о них.
\subsubsection{Требования к способам и средствам обеспечения информационного взаимодействия компонентов}
Взаимодействие между базой данных и API должно происходить путем прямого подключения через драйвер базы данных.

Взаимодействие между модулем рекомендации может происходить как по следующим средствам: брокер сообщений, RESTapi,
внутри программное взаимодействие.

Взаимодействие между API и мобильным приложением должно происходить через RESTapi и веб-сокеты.

\subsubsection{Требования к характеристикам взаимосвязей создаваемой АС со смежными АС}
Смежными системами к данной являются:
\begin{itemize}
    \item PostgresSQL;
    \item Модели по векторизации текстов;
    \item Решающие деревья рекомендации.
\end{itemize}

Данные системы будут подключаемыми программными модулями для API и модуля рекомендации.
Они должны быть совместимы с целевым языком программирования данных систем: Python 3.9

\subsubsection{Требования к режимам функционирования}

У приложения должно быть 2 режима функционирования:
\begin{itemize}
    \item С рекомендательной системой;
    \item С отключенной рекомендательной системой.
\end{itemize}

В связи со специфичностью работы модуля рекомендации разрабатываемой системы.
Возможны случаи когда система не сможет обрабатывать новые запросы по рекомендации в связи с загруженностью системы.
В данном режиме пользователь должен получить оповещение о проблеме и о том что его рекомендации появиться позже.

\subsubsection{Перспективы развития, модернизации АС}
В будущем возможна создание интерфейса для подключения как к другим систем, так и общения других решений с
разрабатываемой системой.

\subsection{Требования к функциям}\label{subsec:func:treb}


\subsubsection{В}

\subsubsection{Регистрация}
Пользователь должен иметь возможность зарегистрироваться в приложении
Для регистрации используются такие данные:
\begin{itemize}
    \item Уникальный логин;
    \item Пароль длиною более 6 символов.
\end{itemize}
Если данный логин уже используется, то пользователь должен быть уведомлен об этом и мобильное приложение должно
попросить ввести новый уникальный логин.

\subsubsection{Авторизация}
Пользователь должен иметь возможность авторизоваться в приложении.
Для авторизации используются такие данные:
\begin{itemize}
    \item Уникальный логин;
    \item Пароль длиною более 6 символов.
\end{itemize}
Если после ввода данных для аутентификации прошло более 30 секунд и не пришел от сервера, пользователь должен быть
оповещен об ошибке на стороне сервера.



\section*{\LARGE Вывод}
\addcontentsline{toc}{section}{Вывод}

В ходе выполнения практической работы были рассмотрены
и проанализированы стандарты из различных категорий.
Эти стандарты охватывают различные аспекты информационных технологий,
управления качеством и информационной безопасности,
что позволило глубже понять особенности и различия между международными,
национальными и межгосударственными стандартами,
а также их роль в разработке программного обеспечения.

Таким образом, анализ вышеупомянутых стандартов
из различных категорий позволил систематизировать знания о стандартизации
в области ИТ и разработки ПО. Выбранные стандарты наглядно демонстрируют,
как различные типы стандартов взаимодействуют
и дополняют друг друга для обеспечения высокого уровня качества,
безопасности и совместимости программных решений.
