\section{Описание предметной области мобильного приложения}

\subsection{Существующие системы рекомендаций}
Большая часть рекомендательных систем уже встроенно в стриминговые сервисы.
Рассмотрим как именно они работают.
\begin{itemize}
    \item Netflix; \par
    Netflix предлагает одну из самых продвинутых систем рекомендаций для фильмов и сериалов.
    Они используют алгоритмы машинного обучения, такие как коллаборативная фильтрация и контентные алгоритмы, чтобы
    адаптировать рекомендации к предпочтениям каждого пользователя.

    \item Amazon Prime Video; \par
    Amazon также использует алгоритмы машинного обучения для предлагания рекомендаций фильмов и телешоу,
    основанных на предпочтениях и поведении пользователей.

    \item IMDb; \par
    IMDb предоставляет оценки и рецензии пользователей, которые могут быть использованы для рекомендаций.
    Они также используют информацию о предпочтениях пользователя и о схожести фильмов для предложения контента.

    \item Okko (ОККО); \par
    Это российская платформа для просмотра фильмов и сериалов онлайн, которая также имеет свою систему рекомендаций.
    Они используют алгоритмы машинного обучения для адаптации предложений под интересы пользователей.

    \item ivi (Иви);\par
    Еще одна популярная российская платформа для просмотра фильмов и сериалов с собственной системой рекомендаций.
    Они также используют алгоритмы машинного обучения для предложения контента пользователям.

    \item КиноПоиск;\par
    Это российский интернет-портал о кино, который предлагает рецензии, рейтинги и рекомендации для фильмов и сериалов.
    Они также используют данные о предпочтениях пользователей и оценках для предложения контента.

\end{itemize}

\paragraph{Netflix}\par
Netflix предлагает одну из самых продвинутых систем рекомендаций для фильмов и сериалов.
Они используют алгоритмы машинного обучения, такие как коллаборативная фильтрация и контентные алгоритмы, чтобы
адаптировать рекомендации к предпочтениям каждого пользователя.
\begin{itemize}
    \item Коллаборативная фильтрация:
    Netflix использует информацию о рейтингах, просмотрах и предпочтениях пользователей для поиска схожих пользователей
    и рекомендации контента, который понравился этим пользователям, но еще не просмотрен другими.
    \item Контентные алгоритмы:
    Алгоритмы анализируют характеристики контента (например, жанр, актеры, режиссеры) и историю просмотров пользователя,
    чтобы предложить фильмы и сериалы, которые могут быть интересны на основе их предпочтений.
    \item Алгоритмы, основанные на поведении пользователя:
    Netflix анализирует поведение пользователя на платформе, такие как остановки, перемотки, добавления в список
    просмотра и т. д., чтобы понять их интересы и предпочтения.
\end{itemize}

\paragraph{Amazon Prime Video}

Amazon также использует алгоритмы машинного обучения для предлагания рекомендаций фильмов и телешоу,
основанных на предпочтениях и поведении пользователей.

\begin{itemize}
    \item Коллаборативная фильтрация:
    Amazon использует данные о рейтингах и просмотрах, чтобы предложить контент, который был популярен среди
    пользователей с похожими вкусами.
    \item Контентные алгоритмы:
    Подобно Netflix, Amazon анализирует характеристики контента и предпочтения пользователей для рекомендации фильмов
    и сериалов.
    \item Алгоритмы, основанные на поведении пользователя:
    Amazon анализирует действия пользователя на платформе для предложения контента, который может быть интересен.
\end{itemize}

\paragraph{Hulu, IMDb, Okko, ivi, КиноПоиск}

Как правило, эти платформы также используют комбинацию коллаборативной фильтрации, контентных алгоритмов и алгоритмов,
основанных на поведении пользователя, для рекомендации контента, который наиболее подходит для каждого пользователя.

\begin{itemize}
    \item IMDb; \par
    IMDb предоставляет оценки и рецензии пользователей, которые могут быть использованы для рекомендаций.
    Они также используют информацию о предпочтениях пользователя и о схожести фильмов для предложения контента.

    \item Okko (ОККО); \par
    Это российская платформа для просмотра фильмов и сериалов онлайн, которая также имеет свою систему рекомендаций.
    Они используют алгоритмы машинного обучения для адаптации предложений под интересы пользователей.

    \item ivi (Иви);\par
    Еще одна популярная российская платформа для просмотра фильмов и сериалов с собственной системой рекомендаций.
    Они также используют алгоритмы машинного обучения для предложения контента пользователям.

    \item КиноПоиск.\par
    Это российский интернет-портал о кино, который предлагает рецензии, рейтинги и рекомендации для фильмов и сериалов.
    Они также используют данные о предпочтениях пользователей и оценках для предложения контента.
\end{itemize}
\clearpage

\section{Обоснование выбора инструментальных средств для разработки мобильного приложения}

\subsection{Обоснование выбора алгоритма реализации системы рекомендаций}
В ходе экспериментов было установлено, что для эффективной работы системы рекомендаций следует применять ансамбль
решающих деревьев и векторизированное описание данных.

Решающие деревья позволяют строить рекомендации на основе оценок, которые пользователи присваивают различным предметам.
Для обучения решающего дерева требуется подготовить данные, разделенные на два класса: 0 и 1.
Класс 0 обозначает, что предмет не понравился пользователю, а класс 1 указывает на то, что предмет понравился.

После завершения процесса обучения решающего дерева, система способна предсказывать оценки для новых данных.
Эти предсказания включают не только класс (0 или 1), но и вероятность, с которой данный класс был предсказан.

Эта вероятность используется для ранжирования предметов в системе рекомендаций.
Чем выше вероятность положительного отклика (класс 1), тем выше предмет будет в списке рекомендаций.
Таким образом, система может предоставлять пользователям более точные и релевантные рекомендации на основе их предыдущих
оценок и предпочтений.


\subsection{Обоснование выбора архетектуры}
В ходе анализа архитектуры было выяснено, что для успешной реализации системы рекомендаций необходимо использовать ряд
специфических инструментов и технологий.
Рассмотрим каждый из них подробно.

Для реализации системы рекомендаций был выбран язык программирования Python.
Этот выбор обусловлен его широкой популярностью, богатым набором библиотек для машинного обучения и анализа данных,
а также активным сообществом разработчиков, что обеспечивает доступ к множеству ресурсов и поддержке.

Для хранения данных о предметах, пользователях и их оценках была выбрана база данных PostgreSQL.
PostgreSQL является мощной реляционной СУБД, которая поддерживает сложные запросы и транзакции,
а также обеспечивает надежность и масштабируемость.
Для работы с этой базой данных в коде Python была выбрана ORM-библиотека SQLAlchemy.
Она позволяет писать запросы к базе данных на языке Python, что упрощает разработку и делает код более читаемым
и поддерживаемым.

Для реализации API, через которое система рекомендаций взаимодействует с внешними приложениями, была выбрана библиотека
FastAPI\@.
FastAPI позволяет создавать высокопроизводительные и масштабируемые API благодаря своей асинхронной природе и простоте
использования.
Она также предоставляет автоматическую документацию, что облегчает интеграцию с другими системами.

В качестве системы развертывания был выбран Docker.
Docker позволяет упаковать приложение и все его зависимости в контейнер, который можно легко развертывать на различных
платформах.
Это обеспечивает консистентность среды выполнения и упрощает процесс развертывания и масштабирования приложения.

Для общения с API со стороны клиента была выбрана библиотека Retrofit для языка программирования Kotlin.
Retrofit предоставляет удобные инструменты для работы с RESTful API, что делает процесс интеграции с сервером быстрым
и надежным.
Kotlin, в свою очередь, используется для разработки приложений под Android, что позволяет легко использовать Retrofit
в мобильных приложениях.

Таким образом, комбинация этих инструментов и технологий обеспечивает надежность, масштабируемость и эффективность
системы рекомендаций на всех этапах ее разработки и эксплуатации.









