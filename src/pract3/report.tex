\section{Составить (нарисовать) путь пользователя в вашей разработке}

Путь пользователя --- это общий алгоритм работы с продуктом. Так
называемый User Flow или путь пользователя, это последовательный
список действий или экранов, по которым может переходить
пользователь в процессе взаимодействия с продуктом.
Как пользователь будет взаимодействовать с продуктом
продемонстрированно на рисунке~\ref{fig:user:flow}.

\begin{image}
	\includegrph[scale= 0.9]{user_road.drawio.png}
	\caption{Путь пользователя в разработке}
	\label{fig:user:flow}
\end{image}

\clearpage

\section{Требования к математическому, программному,
	техническому и информационному обеспечению}

\subsection{Математическое обеспечение}

Система рекомендации должна быть разработанная по данному
принципу: Короткое описание фильма векториризируеться, данный вектор
обогащается дополнительными данными о фильме.
Для рекомендации должен использоваться лес решающих деревьев,
который будет уникальным для каждого пользователя. Он будет
модифицироваться при каждом запросе рекомендации от пользователя.

\subsection{Программное обеспечение}

Операционная система сервера: Linux \par
Операционная система клиента: Android \par
Язык программирования: Python (версии 3.8 и выше).\par

\subsection{Техническое обеспечение}

\textbf{Минимальные системные требования:}

\begin{itemize}
	\item Процессор: 3 ГГц, 6 ядер;
	\item ОЗУ: 16 ГБ;
	\item Дисковое пространство: от 50Гб.
	\item GPU с поддержкой CUDA
\end{itemize}


\subsection{Информационное обеспечение}

В качестве базы данных должна использоваться СУБД PostgresSQL.
Формат обмена сообщений между клиентом Android и API должен быть
описан в Swagger документе.

\section{Перечень документации на программный продукт}

Руководство разработчика:

\begin{itemize}
	\item Архитектура программного продукта;
	\item Описание зависимостей;
	\item Swagger документ обмена сообщений между клиентом Android и API;
	\item Описание кода.
\end{itemize}

Техническое задание:
Описание функциональности программы, требований к ПО
и оборудованию, критериев успешности.

Программа и методика испытаний:
План тестирования продукта,
включая тест-кейсы для различных режимов работы и данных.
\section{Требования к эксплуатации разработки}

\subsection{Правильность}
Рекомендательная система должна строго соответствовать техническому заданию и корректно выполнять все заложенные функции, такие как классификация фильмов на основе их описаний с использованием дерева решения. Тестирование не может гарантировать нахождение всех ошибок, поэтому основное внимание должно уделяться исправлению самых критичных и очевидных дефектов, связанных с рекомендательными алгоритмами, взаимодействием через Rest API и корректной работой Android-приложения.

\subsection{Универсальность}
Система должна корректно функционировать при всех допустимых входных данных, получаемых через интерфейс Android-приложения, а также данные, хранящиеся в базе данных Postgres. Необходимо предусмотреть защиту от некорректных или несанкционированных данных, передаваемых через Rest API. Универсальность системы должна поддерживаться в различных сценариях использования рекомендательной логики.

\subsection{Надежность (помехозащищенность)}
Система должна обеспечивать повторяемость результатов рекомендаций даже в случае сбоев, вызванных как техническими, так и программными проблемами. В случае сбоя приложения или базы данных должны быть предусмотрены механизмы восстановления, такие как резервные копии данных и функции аварийного восстановления, предотвращающие потерю данных и некорректную работу рекомендательных алгоритмов.

\subsection{Проверяемость}
Результаты работы системы должны быть легко проверяемыми. Необходимо сохранять исходные данные, параметры деревьев решений и запросы через Rest API, чтобы они могли быть использованы для проверки и анализа в случае возникновения ошибок или некорректных рекомендаций.

\subsection{Точность результатов}
Система должна обеспечивать точность рекомендаций в соответствии с заданными критериями. Точность зависит от корректности данных, хранящихся в Postgres, адекватности использованной модели дерева решения и корректной обработки запросов и ответов через Rest API.
\section{Программа и методика испытаний}

\subsection{Объект испытаний}

\subsubsection{Наименование системы}

Мобильное приложение для рекомендаций фильмов

\subsubsection{Область применения системы}

Программный продукт представляет собой рекомендательную систему для фильмов, основанную на анализе описаний с использованием дерева решений. Основная цель системы — предложить пользователю персонализированные рекомендации фильмов на основе их интересов и предпочтений. Android-приложение взаимодействует с серверной частью через Rest API, которая обрабатывает запросы, используя данные, хранящиеся в базе данных Postgres.

Основные функции системы включают:

\begin{itemize}
	\item Рекомендация фильмов на основе анализа описаний и предпочтений пользователя с помощью дерева решений.
	\item Взаимодействие с сервером для получения и отправки данных через Rest API.
	\item Хранение и управление данными о фильмах и предпочтениях пользователей в базе данных Postgres.
	\item Предоставление интерфейса для поиска и получения рекомендаций в мобильном приложении.
\end{itemize}

Области применения:

\begin{itemize}
	\item Использование в мобильных приложениях для рекомендаций фильмов, чтобы улучшить пользовательский опыт и предложить фильмы, соответствующие интересам пользователей.
	\item Интеграция в существующие онлайн-платформы и приложения, занимающиеся предоставлением медиа-контента.
	\item Применение для исследовательских целей в области машинного обучения и рекомендательных систем для улучшения точности рекомендаций.
\end{itemize}

\subsubsection{Условное обозначение системы}

Условное обозначение системы — РСФ (Рекомендательная Система Фильмов).

\subsection{Цель испытаний}

Целью проводимых по настоящей программе и методике испытаний РСФ (Рекомендательной Системы Фильмов) является определение функциональной работоспособности системы на этапе проведения испытаний.

Программа испытаний должна удостоверить работоспособность РСФ в соответствии с её функциональным предназначением — предоставление пользователю точных рекомендаций фильмов на основе анализа описаний с использованием дерева решений, корректной работы с базой данных Postgres, а также надёжного взаимодействия через Rest API с мобильным Android-приложением.

\subsection{Общие положения}

\subsubsection{Перечень руководящих документов, на основании которых проводятся испытания}

Приёмочные испытания РСФ (Рекомендательной Системы Фильмов) проводятся на основании следующих документов:

\begin{itemize}
	\item Утверждённое Техническое задание на разработку РСФ;
	\item Настоящая Программа и методика приёмочных испытаний;
\end{itemize}

\subsubsection{Место и продолжительность испытаний}

Место проведения испытаний — площадка Заказчика.
Продолжительность испытаний устанавливается Приказом Заказчика о составе приёмочной комиссии и проведении приёмочных испытаний.

\subsubsection{Организации, участвующие в испытаниях}

В приёмочных испытаниях участвуют представители следующих организаций:

\begin{itemize}
	\item ООО \("\)Кинопоиск\("\) (Заказчик);
	\item Шестаокв Ярослев Евгеньевич (Исполнитель).
\end{itemize}

Конкретный перечень лиц, ответственных за проведение испытаний системы, определяется Заказчиком.

\subsubsection{Перечень предъявляемых на испытания документов}

Для проведения испытаний Исполнителем предъявляются следующие документы:

\begin{itemize}
	\item Техническое задание на создание РСФ;
	\item Технический проект РСФ;
	\item Программа и методика испытаний на РСФ;
\end{itemize}
\subsection{Объём испытаний}

\subsubsection{Перечень этапов испытаний и проверок}

В процессе проведения приёмочных испытаний должны быть протестированы следующие подсистемы РСФ (Рекомендательной Системы Фильмов):

\begin{itemize}
	\item Подсистема обработки пользовательского ввода в Android-приложении;
	\item Подсистема анализа данных фильмов с использованием дерева решений;
	\item Подсистема взаимодействия с базой данных Postgres;
	\item Подсистема Rest API для обработки запросов;
	\item Подсистема генерации рекомендаций.
\end{itemize}

Все подсистемы испытываются одновременно на корректность взаимодействия, влияние одной подсистемы на другие, то есть испытания проводятся комплексно.

Приёмочные испытания включают проверку:

\begin{itemize}
	\item полноты и качества реализации функций, указанных в ТЗ;
	\item выполнения каждого требования, относящегося к интерфейсу Android-приложения;
	\item работы пользователей в диалоговом режиме;
	\item полноты действий, доступных пользователю, и их достаточности для функционирования системы;
	\item простоты использования приложения, возможности работы пользователей без специальной подготовки;
	\item реакции системы на ошибки пользователя;
	\item практической выполнимости рекомендованных процедур.
\end{itemize}

\subsubsection{Испытания подсистемы обработки пользовательского ввода}

Испытания подсистемы обработки пользовательского ввода направлены на проверку доступности всех функций приложения, удобства использования интерфейса, корректного ввода данных и обработки ошибок. Также проверяется устойчивость приложения при сбоях и стабильность работы при различных сценариях пользовательского ввода.

\subsubsection{Испытания подсистемы анализа данных фильмов}

Тестируется корректность работы дерева решений для генерации рекомендаций на основе описаний фильмов и пользовательских предпочтений. Проверяется точность модели, реакция на некорректные или неполные данные, а также производительность при больших объёмах данных.

\subsubsection{Испытания подсистемы взаимодействия с базой данных Postgres}

Проверяется правильность записи и чтения данных из базы данных Postgres, включая корректную обработку запросов, отказоустойчивость, а также производительность при большом количестве запросов и данных.

\subsubsection{Испытания подсистемы Rest API}

Основная цель — убедиться в правильности обработки запросов через Rest API, включая генерацию корректных ответов, обработку ошибок, а также устойчивость при высокой нагрузке и сбоях. Тестируется производительность и надёжность обмена данными между клиентом и сервером.

\subsubsection{Испытания подсистемы генерации рекомендаций}

Тестируется полнота и точность генерируемых рекомендаций, проверяется их соответствие интересам пользователя. Также тестируется производительность подсистемы и её способность обрабатывать запросы при большом количестве пользователей.

\subsection{Методика проведения испытаний}

\begin{longtable}{|c|p{7.5cm}|p{7.5cm}|}
	\caption{\leftline{Методика проведения испытаний}} \label{table:test} \\
	\hline
	\textbf{\No} & \textbf{Действие} & \textbf{Результат} \\
	\hline
	\endfirsthead
	\conttable{table:test} \\
	\hline
	\textbf{\No} & \textbf{Действие} & \textbf{Результат} \\
	\hline
	\endhead

	\textbf{1}
	& \multicolumn{2}{|l|}{\textbf{Сценарий <<Тестирование обработки пользовательского ввода>>}} \\ \hline
	1.1
	& Ввести некорректные данные (например, неверный запрос в Android-приложении) и попытаться получить рекомендации.
	& Отображается сообщение об ошибке, приложение не завершает работу аварийно. \\ \hline

	1.2
	& Ввести корректные данные, запросить рекомендации по фильму.
	& Запрос успешно обработан, отображаются результаты рекомендаций. \\ \hline

	1.3
	& Проверить лог действий в приложении.
	& Лог корректно отображает все этапы обработки запроса. \\ \hline

	\textbf{2}
	& \multicolumn{2}{|l|}{\textbf{Сценарий <<Тестирование анализа данных фильмов>>}} \\ \hline
	2.1
	& Передать описание фильма с неполными данными в подсистему анализа.
	& Программа корректно обрабатывает данные, генерирует рекомендации на основе доступной информации. \\ \hline

	2.2
	& Передать большое количество описаний фильмов и запросить рекомендации.
	& Подсистема успешно обрабатывает данные, время выполнения соответствует ожидаемым параметрам. \\ \hline

	\textbf{3}
	& \multicolumn{2}{|l|}{\textbf{Сценарий <<Тестирование взаимодействия с базой данных Postgres>>}} \\ \hline
	3.1
	& Выполнить запрос на получение данных о фильмах из базы Postgres.
	& Данные успешно получены, отображаются корректно в приложении. \\ \hline

	3.2
	& Выполнить запрос с большими объемами данных (1000+ фильмов).
	& Запрос выполнен без ошибок, производительность соответствует заявленным требованиям. \\ \hline

	\textbf{4}
	& \multicolumn{2}{|l|}{\textbf{Сценарий <<Тестирование Rest API>>}} \\ \hline
	4.1
	& Выполнить запрос на получение рекомендаций через Rest API.
	& API возвращает корректный результат, ошибки отсутствуют. \\ \hline

	4.2
	& Выполнить тестирование на высокую нагрузку (100+ параллельных запросов).
	& Rest API выдерживает нагрузку, время ответа в пределах допустимых значений. \\ \hline

	\textbf{5}
	& \multicolumn{2}{|l|}{\textbf{Сценарий <<Тестирование подсистемы генерации рекомендаций>>}} \\ \hline
	5.1
	& Запросить рекомендации на основе описания фильма с различными жанрами и параметрами.
	& Система корректно генерирует рекомендации в соответствии с предпочтениями пользователя. \\ \hline

	5.2
	& Проверить корректность рекомендаций на основе пользовательских оценок.
	& Рекомендации соответствуют предпочтениям и параметрам, указанным пользователем. \\ \hline
\end{longtable}

\subsection{Требования по испытаниям программных средств}

Испытания программных средств РСФ проводятся в процессе функционального тестирования системы и нагрузочного тестирования. Других требований по испытаниям программных средств РСФ не предъявляется.

\subsection{Перечень работ, проводимых после завершения испытаний}

По результатам испытаний составляется заключение о соответствии РСФ требованиям ТЗ и возможности оформления акта сдачи в эксплуатацию. При необходимости проводится доработка системы и документации.

\subsection{Условия и порядок проведения испытаний}

Испытания РСФ проводятся на оборудовании Заказчика с использованием Android-устройств и серверов для хостинга базы данных и Rest API. Во время испытаний проводится полное функциональное тестирование в соответствии с требованиями ТЗ.

\subsection{Материально-техническое обеспечение испытаний}

Испытания проводятся на оборудовании Заказчика с следующей минимальной конфигурацией:

\begin{itemize}
	\item Android-устройство для тестирования приложения;
	\item Сервер с установленной базой данных Postgres;
	\item Сервер для Rest API;
	\item Операционная система: Linux.
\end{itemize}

\subsection{Метрологическое обеспечение испытаний}

Испытания РСФ не требуют использования специализированного измерительного оборудования.

\subsection{Отчётность}

Результаты испытаний фиксируются в протоколах, содержащих информацию о выполненных тестах, используемых технических средствах, методиках проведения тестирования и оценке результатов. В конце испытаний оформляется акт сдачи РСФ в эксплуатацию.

\clearpage

\section*{\LARGE Вывод} \addcontentsline{toc}{section}{Вывод}

В ходе практической работы была разработана структура и методология для создания программного продукта — системы рекомендаций фильмов, которая основана на анализе пользовательских предпочтений и фильмов. Эта система включает Android-приложение, серверную часть с Rest API и базу данных Postgres для хранения информации о фильмах и пользователях.\par

Проведённая работа позволила получить практический опыт в разработке программного обеспечения от начального этапа проектирования до тестирования и документирования. Были разработаны и протестированы основные компоненты системы, включая механизм генерации рекомендаций, взаимодействие с базой данных и корректную обработку пользовательского ввода.

Результатом работы стала функциональная система рекомендаций фильмов, готовая к использованию и дальнейшему развитию. Работа помогла глубже понять все этапы разработки, от постановки задач до их реализации, и позволила приобрести ценные навыки программирования, документирования и тестирования, что является важным аспектом в профессиональной разработке программного обеспечения.