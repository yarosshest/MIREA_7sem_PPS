\section{Диаграмма прецедентов}

Диаграмма прецедентов (Use case diagram) --- это диаграмма поведения,
на которой показано множество прецедентов и актёров,
а также отношения между ними.\par
Диаграммы прецедентов применяются для моделирования вида системы
с точки зрения внешнего наблюдателя.\par
Основные элементы диаграммы прецедентов:

\begin{itemize}
	\item Субъект (actor) --- любая сущность,
		взаимодействующая с системой извне.
	\item Прецеденты (use case) --- описание множества последовательностей
		действий (включая их варианты),
		которые выполняются системой для того,
		чтобы актёр получил результат,
		имеющий для него определённое значение.
\end{itemize}

Между субъектами и прецедентами могут существовать различные отношения,
которые описывают взаимодействие экземпляров одних субъектов
и прецедентов с экземплярами других субъектов и прецедентов.

\begin{image}
	\includegrph[scale=0.27]{usecse}
	\caption{Диаграмма прецедентов}
	\label{fig:use:case}
\end{image}

\section{Диаграмма последовательности}

Диаграмма последовательности (sequence diagram) --- это наглядное
представление совокупности разных элементов модели системы,
изображение того, как и в каком порядке они взаимодействуют.\par
Такие диаграммы подробно описывают, как выполняются разные операции.
При этом они показывают временной порядок или хронологию:
то, когда, как и в какой очереди передаются сообщения.\par
Диаграммы удобно использовать при проектировании или проверке архитектуры,
логики системы или интерфейса.

\begin{image}
	\includegrph[scale=0.27]{sec}
	\caption{Диаграмма последовательности}
	\label{fig:sequence}
\end{image}

\section{Паттерн проектирования}

Для разработки системы, которая управляет продуктами, их характеристиками и рекомендациями, выбран паттерн \textbf{<<Фабрика>>} (Factory Method).\par
Паттерн <<Фабрика>> используется для создания объектов без указания точного типа создаваемого объекта. Это позволяет упростить процесс создания различных типов объектов в системе, таких как \textbf{Product}, \textbf{Rate}, \textbf{Attribute}, \textbf{Vector} и другие, обеспечивая гибкость и модульность системы.\par
Каждый объект (например, продукт, оценка или атрибут) создается с помощью отдельной фабрики, которая инкапсулирует детали создания объектов и позволяет добавлять новые типы объектов без изменения существующего кода.\par
Этот паттерн идеален для систем, где необходимо часто создавать разнообразные объекты с различными параметрами и характеристиками, такие как описание продукта, его атрибуты, леммы и другие параметры.\par
Использование паттерна <<Фабрика>> также позволяет эффективно управлять зависимостями и упрощает тестирование, так как объекты создаются через фабричные методы, а не напрямую, что облегчает внедрение мок-объектов или подделок для тестов.\par
Таким образом, паттерн <<Фабрика>> подходит для разработки системы, обеспечивая гибкость в создании объектов, упрощая добавление новых типов объектов и повышая масштабируемость и модульность всей системы.


\clearpage

\section*{\LARGE Вывод}
\addcontentsline{toc}{section}{Вывод}

В результате работы были определены ключевые аспекты проектирования системы,
включая диаграммы прецедентов и последовательностей,
выбор паттерна проектирования.

